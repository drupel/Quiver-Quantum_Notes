\documentclass{book}
\usepackage{amsmath,amssymb,latexsym,mathtools,tikz-cd,amsthm,stackrel}
\usepackage[margin=1in]{geometry}

\title{From Quivers to Quantum Groups}
\author{Dylan Rupel}

% Define theorem-like environments
 \newtheorem{theorem}{Theorem}[section]
 \newtheorem{lemma}[theorem]{Lemma}
 \newtheorem{corollary}[theorem]{Corollary}
 \newtheorem{prop}[theorem]{Proposition}
 \newtheorem{definition}[theorem]{Definition}
 \newtheorem{remark}[theorem]{Remark}
 \newtheorem{example}{Example}[section]
 \newtheorem {exercise}[theorem] {Exercise}
 \newtheorem* {obs*}{Observation}
 \newtheorem* {recall*}{Recall}
 % Define math operators
 \DeclareMathOperator{\rank}{rank}
 \DeclareMathOperator{\Hom}{Hom}
 \DeclareMathOperator{\Rep}{Rep}
 \DeclareMathOperator{\rep}{rep}
 \DeclareMathOperator{\im}{im}
 \DeclareMathOperator{\coker}{coker}
 \newcommand{\id}{\mathrm{id}}

\begin{document}
  \maketitle

%%%%%%%%%%%%%%%%%%%%%%
\chapter{Introduction}

  These are notes for a graduate course at University of Notre Dame, Spring 2018.

  \section{Acknowledgements}

%%%%%%%%%%%%%%%%%%%%%%%%%%%%%%%%
\chapter{Quiver Representations}

  \section{Motivation: Classification Problems in Linear Algebra}
  
  Study configurations of vector spaces and linear maps up to simultaneous base change.
  
  	\subsection{Baby Examples}
  		\begin{enumerate}
			\item Vector spaces: completely classified by dimension. $V \cong \mathbb{C}^n$ for some $n$.
			\item A single linear map: $f:U\to V$. There are three invariants: $\dim U, \dim V, \rank f$. Up to base change, $f$ can be represented uniquely in a left-justified reduced row-echelon form.
			\item Endomorphism: $f:V\to V$. Have to fix $\dim V=n$. How to classify $n\times n$ matrices up to conjugation? Use Jordan Canonical Form: $f$ is completely determined by its eigenvalues and sizes/multiplicities of its Jordan blocks. 
  		\end{enumerate}

  	\subsection{Subspace Problems}
		\begin{enumerate}
			\item One Subspace Problem: How to classify $U \subseteq V$? Fix $\dim U$ and $\dim V$. By changing bases in $V$, a subspace can be moved to any other subspace of the same dimension.
			\item Two Subspace Problem:  $U_1,U_2 \subseteq V$. Fix $\dim U_i, \dim V,$ and $\dim (U_1\cap U_2)$. This is enough to classify such configurations.
			\item Three Subspace Problem: $U_1,U_2,U_3 \subseteq V$. How to classify ordered triples of subspaces in a fixed vector space up to simultaneous base change?  Fix $\dim U_i, \dim V, \dim (U_i\cap U_j),$ and $\dim (U_1\cap U_2\cap U_3)$. This alone is not enough.
				\begin{example}
					Let $\dim V=3$ and $\dim U_i=1$. Consider \[V=\mathbb{C}^3, U_1 = \mathbb{C} \begin{bmatrix}1\\0\\0\end{bmatrix},U_2 = \mathbb{C} \begin{bmatrix}0\\1\\0\end{bmatrix},U_3 = \mathbb{C} \begin{bmatrix}0\\0\\1\end{bmatrix}\] and  \[V=\mathbb{C}^3, U_1 = \mathbb{C} \begin{bmatrix}1\\0\\0\end{bmatrix},U_2 = \mathbb{C} \begin{bmatrix}0\\1\\0\end{bmatrix},U_3 = \mathbb{C} \begin{bmatrix}1\\1\\0\end{bmatrix}.\] These are not equivalent.
				\end{example}
			So we must also fix $\dim ((U_i + U_j)\cap U_k)$. This turns out to be enough.
      			\item Four Subspace Problem: $U_1,U_2,U_3, U_4 \subseteq V$. Fix $\dim U_i, \dim V, \dim (U_i\cap U_j),\dim (U_i\cap U_j\cap U_k), \dim ((U_i + U_j)\cap U_k),\dim (U_1\cap U_2 \cap U_3 \cap U_4),  \dim ((U_i + U_j+U_k)\cap U_l),$ and $ \dim ((U_i+ U_j)\cap (U_k+U_l)).$ This is not enough.
				\begin{example}
					Let $\dim V=2$, and consider \[V=\mathbb{C}^2, U_1 = \mathbb{C} \begin{bmatrix}1\\0\end{bmatrix},U_2 = \mathbb{C} \begin{bmatrix}0\\1\end{bmatrix},U_3 = \mathbb{C} \begin{bmatrix}1\\1\end{bmatrix},U_4 = \mathbb{C} \begin{bmatrix}1\\ \lambda\end{bmatrix}\] where $\lambda\neq 0,1$. We need to know this continuous parameter $\lambda$ to distinguish such configurations. 
				\end{example}
			More generally: the endomorphism problem embeds into the four subspace problem. For $f:U\to U$, we consider \[V=U\oplus U, \ U_1=U\oplus 0, \ U_2=0\oplus U, \ U_3=\{(x,x):x\in U\},\ U_4=\{(x,f(x)):x\in U\}.\] A base change in $V$ fixing $U_1,U_2,U_3$ amounts to conjugation of $f$. 
			\item Five Subspace Problem: Hopeless...
		\end{enumerate}
		

  \section{Basic Definitions}
  
	\begin{definition} A \textbf{quiver} $Q=(Q_0,Q_1,s,t)$ consists of a set $Q_0$ of vertices, a set $Q_1$ of arrows, and maps $s,t:Q_1\to Q_0$. For $\alpha\in Q_1$, we write $\alpha:s(\alpha)\to t(\alpha).$
 	 \end{definition}
  
  	\begin{definition} Fix a field $\mathbb{F}$. An \textbf{$\mathbb{F}$-representation} of $Q$ is a set \[V= \left(\{V_i\}_{i\in Q_0},\{V_\alpha\}_{\alpha\in Q_1} \right)\] where $V_i$ is an $\mathbb{F}$-vector space and $V_\alpha : V_{s(\alpha)}\to V_{t(\alpha)}$ is an $\mathbb{F}$-linear map.
   	 \end{definition}
    
  	\begin{example}
    	Let $Q$ be the quiver $1\rightarrow 3\leftarrow 2$ with $Q_0=\{1,2,3\}$. A representation $V$ can be drawn as $V_1 \xrightarrow{f_1} V_3 \xleftarrow{f_2} V_2$. For $f_i$ injective, this gives an instance of the two subspace problem. Changing bases in $V_1,V_2,V_3$ amounts to conjugating each $f_i$ by automorphisms of the $V_i$, say $g_i:V_i\to V_i$. That is, the configuration $V_1 \xrightarrow{g_3 f_1 g_1^{-1}} V_3 \xleftarrow{g_3 f_2 g_2^{-1}} V_2$ is equivalent to the original. Putting all these together, 
		\[ \begin{tikzcd}
		V_1 \arrow{r}{f_1} \arrow{d}{g_1} & V_3 \arrow{d}{g_3}& V_2 \arrow{l}{f_2}\arrow{d}{g_2} \\
		V_1 \arrow{r}{g_3f_1g_1^{-1}}& V_3 & V_2 \arrow{l}{g_3f_2g_2^{-1}}
		\end{tikzcd}\]
	where each square commutes.
    	\end{example}
    
   	 \begin{definition}
    	A \textbf{morphism} $\theta:V\to W$ between representations $V$ and $W$ of $Q$ consists of $\mathbb{F}$-linear maps $\theta_i:V_i\to W_i$ such that $\theta_{t(\alpha)} \circ V_\alpha = W_\alpha\circ \theta_{s(\alpha)}$ for all $\alpha\in Q_1$. That is,
	\[ \begin{tikzcd}
		V_{s(\alpha)} \arrow{r}{V_\alpha} \arrow{d}{\theta_{s(\alpha)}} & V_{t(\alpha)} \arrow{d}{\theta_{t(\alpha)}}\\
		W_{s(\alpha)} \arrow{r}{W_\alpha}& W_{t(\alpha)}
		\end{tikzcd}\]
		commutes for every $\alpha\in Q_1$.
    	\end{definition}
    
   	 \begin{definition}
    	Let $\Rep_\mathbb{F}Q$ be the category of all $\mathbb{F}$-representations of $Q$. (Here, the identity maps of representations have all components the identity maps on the $V_i$.) Write $\rep_\mathbb{F}Q \subseteq \Rep_\mathbb{F}Q$ for the full subcategory consisting of representations with finite dimensional spaces $V_i$.
    	\end{definition}
    
    	\begin{lemma}
    		An isomorphism in $\Rep_\mathbb{F}Q$ is any map $\theta:V\to W$ with each $\theta_i:V_i\to W_i$ invertible.
    	\end{lemma}
    
    	Thus each classification problem is the study of isomorphism classes in $\Rep_\mathbb{F}Q$ for some quiver $Q$.

	\begin{definition}
		Given $v\in \rep_\mathbb{F}Q$, write $\underline{\dim} V = (\dim V_i)_{i\in Q_0}\in \mathbb{N}^{Q_0}$ for the \textbf{dimension vector} of $V$.
	\end{definition}
	
	Fix a dimension vector $\underline{d} \in \mathbb{N}^{Q_0}$. Consider the affine space $M_{\underline{d}}(Q)=\bigoplus_{\alpha\in Q_1} \Hom_{\mathbb{F}}\left(\mathbb{F}^{d_{s(\alpha)}},\mathbb{F}^{d_{t(\alpha)}}\right)$. This consists of all representations of $Q$ of a fixed dimension vector. The group $G_{\underline{d}}= \prod_{i\in Q_0} GL_{d_i}(\mathbb{F})$ acts on $M_{\underline{d}}(Q)$ by $g\cdot (f_\alpha)_{\alpha\in Q_1}=\left(g_{t(\alpha)}f_\alpha g_{s(\alpha)}^{-1}\right)_{\alpha\in Q_1}.$
	
	\begin{lemma}
		For any $g\in G_{\underline{d}}$ and $(f_\alpha)_{\alpha\in Q_1}$, the representations corresponding to $(f_\alpha)_{\alpha\in Q_1}$ and  $g\cdot (f_\alpha)_{\alpha\in Q_1}$ are isomorphic.
	\end{lemma}
	
	Thus the classification problem amounts to understanding the $G_{\underline{d}}$-orbit structure of $M_{\underline{d}}(Q)$.
	
	\subsection{Properties of $\Rep_\mathbb{F}Q$}
		$Rep_\mathbb{F}Q$ is an additive category: For $V,W\in \Rep_\mathbb{F}Q$, write $\Hom_Q(V,W)$ for the set of morphisms from $V$ to $W$. For $\theta,\theta' \in \Hom_Q(V,W)$ we have another morphism $\theta+\theta'$ with $(\theta+\theta')_i=\theta_i+\theta'_i$. This gives $\Hom_Q(V,W)$ the structure of an abelian group. 
		
		\begin{definition}
			Given $V,W\in \Rep_\mathbb{F}Q$ define $V\oplus W$ to be the representation with $(V\oplus W)_i=V_i\oplus W_i$ and $(V\oplus W)_\alpha = V_\alpha \oplus W_\alpha$. 
		\end{definition}
		
		\begin{definition}
			A representation $V \in \Rep_\mathbb{F}Q$ is \textbf{indecomposable} if it can't be written as $U\oplus W$ for any nontrivial representations $U,W$.
		\end{definition}
		
		\begin{example}
			Let $Q: \ 1\to3\leftarrow 2$. Let $V_1 \xrightarrow{f_1} V_3 \xleftarrow{f_2} V_2$ be a representation of $Q$. Considering $\ker f_1$ and $\ker f_2$, we get a direct sum decomposition where  $$\mathbb{F}^{\dim(\ker f_1)} \rightarrow 0 \leftarrow 0$$ and $$ 0\rightarrow 0 \leftarrow \mathbb{F}^{\dim(\ker f_2)}$$ split off. Thus we can assume $f_1$ and $f_2$ are both injective. Considering $\im f_1\cap \im f_2 \subseteq V_3$, we get a summand of the form \[(\mathbb{F}\rightarrow\mathbb{F}\leftarrow\mathbb{F})^{\dim(\im f_1\cap \im f_2)}.\] Then we can assume $\im f_1\cap \im f_2=0$. Then we get summands of the form 
			\begin{align*} V_1\rightarrow\im f_1 \leftarrow 0 &\cong (\mathbb{F}\rightarrow\mathbb{F}\leftarrow 0)^{\dim V_1} \\ 0\rightarrow\im f_2 \leftarrow V_2 &\cong (0 \rightarrow\mathbb{F}\leftarrow\mathbb{F})^{\dim V_2} \end{align*}
			Thus we may assume $V_1$ and $V_2$ are zero and we get a summand \[(0 \rightarrow\mathbb{F}\leftarrow0)^{\dim V_3}. \]
		\end{example}
		
		\begin{definition}
			Given $\theta:V\to W$, define $K=(K_i,K_\alpha)$ where $K_i=\ker\theta_i$ and \[ \begin{tikzcd}
		K_{s(\alpha)} \arrow[dashed]{r}{K_\alpha} \arrow{d} & K_{t(\alpha)} \arrow{d}\\
		V_{s(\alpha)} \arrow{r}{V_\alpha} \arrow{d}{\theta_{s(\alpha)}} & V_{t(\alpha)} \arrow{d}{\theta_{t(\alpha)}}\\
		W_{s(\alpha)} \arrow{r}{W_\alpha}& W_{t(\alpha)}
		\end{tikzcd}\]
		\end{definition}
		
		
		\begin{lemma}
			The representation $K$ is the kernel of $\theta:V\to W$ in the category $\Rep_\mathbb{F}Q$.
		\end{lemma}
		
		\begin{proof}
		    Suppose $\varphi:x\to V$ is a morphism of representations satisfying $\theta\circ\varphi=0$. We need to show there exists a unique map $\psi:X\to K$ so that the diagram 
		    \[ \begin{tikzcd}
		    & X \arrow[dashed]{dl} \arrow{d}{\varphi}&\\
		    K\arrow[hookrightarrow]{r} & V \arrow{r}{\theta} & W 
		    \end{tikzcd}\]
		    commutes. For any $x_i\in X_i$, we have $\theta_i(\varphi_i(x_i))=0$. So $\varphi_i(x_i)\in\ker\theta_i$ and we must define $\psi_i(x_i)=\varphi_i(x_i)$. To see that these give a morphism of representations, we check for $\alpha\in Q_1$ and $x_{s(\alpha)}\in X_{s(\alpha)}$:
		    \[K_\alpha\circ\psi_{s(\alpha)}(x_{s(\alpha)})=V_\alpha\circ\varphi_{s(\alpha)}(x_{s(\alpha)})=\varphi_{t(\alpha)}\circ X_\alpha(x_{s(\alpha)})=\psi_{t(\alpha)}\circ X_\alpha(x_{s(\alpha)}).\]
		\end{proof}
		
        \begin{definition}
            Given $\theta:V\to W$, define a representation $C$ with $C_i=\coker\theta_i$ for $i\in Q_0$ and $C_\alpha$ for $\alpha\in Q_1$ as the map induced by $W_\alpha$ in this diagram:
            \[ \begin{tikzcd}
	    	V_{s(\alpha)} \arrow{r}{V_\alpha} \arrow{d}{\theta_{s(\alpha)}} & V_{t(\alpha)} \arrow{d}{\theta_{t(\alpha)}}\\
	    	W_{s(\alpha)} \arrow{d} \arrow{r}{W_\alpha}& W_{t(\alpha)}\arrow{d}\\
	    	C_{s(\alpha)} \arrow[dashed]{r}{C_\alpha}  & C_{t(\alpha)}
	    	\end{tikzcd}\]
        \end{definition}
        
        \begin{exercise}
            Prove that $C$ is the cokernel of the morphism $\theta:V\to W$ in the category $\Rep_\mathbb{F}Q$. \\ Hint: Show that dualizing all vector spaces and linear maps gives a contravariant functor $\rep_\mathbb{F}Q \to \rep_\mathbb{F}Q^{op}$ which takes cokernels in $\rep_\mathbb{F}Q$ to kernels in $\rep_\mathbb{F}Q^{op}$.
        \end{exercise}

        \begin{definition}
            A representation $U$ is called a \textbf{subrepresentation} of a representation $V$ if there exists a monomorphism of representations $\theta: U\to V$, i.e. each $\theta_i$ is an injective linear map. Define $V/U$ as the cokernel of $\theta$ in this case.
        \end{definition}
        
        \begin{theorem}[First Isomorphism Theorem]
            For any morphism $\theta: V\to W$ of representations, we have an isomorphism $\im\theta\cong V / \ker\theta$.
        \end{theorem}
        
        \begin{proof}
            Observe that $\im\theta=(\theta_i(v_i),I_\alpha)$ where for $v_{s(\alpha)}\in V_{s(\alpha)}$ we get \[I_\alpha(\theta_{s(\alpha)}(v_{s(\alpha)}))=\theta_{t(\alpha)}(V_\alpha(v_{s(\alpha)}))\] and $V/\ker\theta=(V_i+\ker\theta_i,Q_\alpha)$ where \[Q_\alpha(v_{s(\alpha)}+\ker\theta_{s(\alpha)})=V_\alpha(v_{s(\alpha)})+\ker\theta_{t(\alpha)}.\] But each $\theta_i$ is $\mathbb{F}$-linear and so there are isomorphisms 
                \begin{align*}
                    \bar{\theta}_i:V_i/\ker\theta_i &\to \im\theta_i \\
                    v_i+\ker\theta_i &\mapsto \theta_i(v_i).
                \end{align*}
            These satisfy $\bar{\theta}_{t(\alpha)}\circ Q_\alpha=I_\alpha\circ\bar{\theta}_{s(\alpha)}$ and so we get an isomorphism $V/\ker\theta\cong\im\theta$.
        \end{proof}
        
        \begin{theorem}
            Let $\mathbb{F}$ be a field and $Q$ a quiver. With the above notions of kernel and cokernel, $\Rep_\mathbb{F}Q$ has the structure of an $\mathbb{F}$-linear abelian category.
        \end{theorem}
        
        \begin{proof}
            Last time we observed $\mathbb{F}$-linearity and additive structure. 
                \begin{exercise}
                    Check that $\Hom()\times\Hom()\to\Hom()$ is bilinear.
                \end{exercise}
            It remains to observe that given a morphism $\theta:V\to W$ with kernel $\iota:K\to V$ and cokernel $\pi:W\to C$, we have $\coker\iota=\ker\theta.$ But this is essentially the statement of the First Isomorphism Theorem.
        \end{proof}
        
        \begin{example}
            We saw last time that the two subspace quiver $1\to 3 \leftarrow 2$ has six isomorphism classes of indecomposables:
                \begin{align*}
                    S_1 &= \mathbb{F} \to 0 \leftarrow 0 \\
                    S_3 &= 0 \to \mathbb{F} \leftarrow 0 \\
                    S_2 &= 0 \to 0 \leftarrow \mathbb{F} \\
                    P_1 &= \mathbb{F} \xrightarrow{\id} \mathbb{F} \leftarrow 0 \\
                    P_2 &= 0 \to \mathbb{F} \xleftarrow{\id} \mathbb{F} \\
                    I_3 &= \mathbb{F} \xrightarrow{\id} \mathbb{F} \xleftarrow{\id} \mathbb{F}
                \end{align*}
            Consider $x\in \Hom_Q(S_3,I_3).$ 
                \[ \begin{tikzcd}
		            0 \arrow{r}{0} \arrow{d}{0} & \mathbb{F} \arrow{d}{x\neq 0} & 0 \arrow{l}{0}\arrow{d}{0}\\
		            \mathbb{F} \arrow{r} \arrow{d} & \mathbb{F}\arrow{d} & \arrow{l} \arrow{d}\mathbb{F} \\
		            \mathbb{F} \arrow{r} & 0 & \mathbb{F} \arrow{l}
		        \end{tikzcd}\]
		    So $\Hom_Q(S_3,I_3)=\mathbb{F}$ and $\coker(x)\cong S_1\oplus S_2.$ In a similar way, we determine that $\Hom_Q(I_3,P_1)=0$ using the diagram below. 
		        \[ \begin{tikzcd}
		            \mathbb{F} \arrow{r} \arrow{d}{0} & \mathbb{F} \arrow{d}{0} & \mathbb{F} \arrow{l}{\id}\arrow{d}{0}\\
		            \mathbb{F} \arrow{r} & \mathbb{F} & 0 \arrow{l}
		        \end{tikzcd}\]
		    Considering all possible homomorphisms between indecomposable representations, we can construct what is called the Auslander-Reiten Quiver for our example:
		        \[\begin{tikzcd}
		            & P_1 \arrow{dr} & & S_2 \\
		            S_3 \arrow{ur}\arrow{dr} && I_3 \arrow{ur}\arrow{dr} \\
		            & P_2 \arrow{ur} && S_1
		        \end{tikzcd}\]
        \end{example}




	\subsection{Extensions}
		\begin{definition} 
			Let U,V, and W be representations of a quiver Q. We call V an \underline{extension of W by U} if there exists an exact sequence $0 \rightarrow U \xrightarrow{\phi} V 				\xrightarrow{\psi} W \rightarrow 0$. \\ i.e. $\phi$ is injective, $\psi$ is surjective, and $\im \phi = \ker \psi$.
		\end{definition}
	We call two extensions $V$ and $V'$ of $W$ by $U$ equivalent if there exists an isomorphism $\theta : V \rightarrow V'$ making the following diagram commute.
		\[ \begin{tikzcd}
		0 \arrow{r} & U \arrow[equal]{d} \arrow{r}{\phi}& V \arrow{r}{\psi}\arrow{d}{\theta} & W  \arrow[equal]{d} \arrow{r} & 0 \\
		0 \arrow{r} & U \arrow{r}{\phi '} & V' \arrow{r}{\psi '} & W \arrow{r}& 0 
		\end{tikzcd}\]
	This clearly defines an equivalence relation on the set of extensions of $W$ by $U$, and we write $ Ext_Q(W,U)$ for the set of equivalences classes of these extensions. 

		\begin{exercise}
			Consider a commutative diagram 
			\[ \begin{tikzcd}
			0 \arrow{r} & U \arrow{d}{\phi} \arrow{r}& V \arrow{r}\arrow{d} & W  \arrow{d}{\psi} \arrow{r} & 0 \\
			0 \arrow{r} & U' \arrow{r} & V' \arrow{r} & W' \arrow{r}& 0 
			\end{tikzcd}\]
			Show that the right hand square is a pullback if and only if $\phi$ is an isomorphism, and show that the left hand square is a pushout if and only if $\psi$ is an isomorphism.
		\end{exercise}
		\begin{definition} 
			Given $[V],[V'] \in Ext_Q(W,U)$, define their \underline{Baer sum} $[V]+[V']=[V'']$ by the following commutative diagram:
			\[ \begin{tikzcd}
			0 \arrow{r} & U \oplus U \arrow[equal]{d} \arrow{r}& V \oplus V' \arrow{r} & W \oplus W \arrow{r}  \arrow[dl, phantom, "\urcorner", very near start] & 0 \\
			0 \arrow{r} & U \oplus U  \arrow[dr, phantom, "\ulcorner", very near start] \arrow[d, "(1 \text{,}-1)"'] \arrow{r} & X \arrow{r}\arrow{u} \arrow{d} & W \arrow[u, "\Delta"'] \arrow{r}\arrow[equal]{d}& 0 \\
			0 \arrow{r} & U \arrow{r} & V'' \arrow{r} & W \arrow{r}& 0 
			\end{tikzcd}\]
		\end{definition}
	It is immediate from the universal property of the pullback that equivalent choices of $V$ and $V'$ produce equivalent $X's$ and then the universal property of the pushout shows that the class of $V''$ does not depend on the choice of representatives for $[V]$ and $[V']$. 

		\begin{prop}
			The Baer sum defines an abelian group structure on the set $Ext_Q(W,U)$ for any $U,W \in Rep_\mathbb{F}Q$ where $[W\oplus U]$ is the identity element and given $[V] \in Ext_Q(W,U)$ represented by a sequence $0 \rightarrow U \xrightarrow{\phi} V \xrightarrow{\psi} W \rightarrow 0$, the class $-[V]$ is represented by the sequence $0 \rightarrow U \xrightarrow{-\phi} V \xrightarrow{\psi} W \rightarrow 0$ or equivalently $0 \rightarrow U \xrightarrow{\phi} V \xrightarrow{-\psi} W \rightarrow 0$

		\end{prop}

		\begin{obs*}
			Let $U$ and $W$ be representations of $Q$. For any choice of maps $g_{\alpha} :W_{s(\alpha)} \rightarrow U_{t(\alpha)}$ for $\alpha \in Q_1$, we can build an extension $V_g$ of $W$ by $U$ where $V_i = W_i \oplus U_i$ for $i \in Q_0$ and $V_{\alpha} = \begin{bmatrix}W_{\alpha} & 0\\ g_{\alpha} & U_{\alpha} \end{bmatrix}$ for $\alpha \in Q_1$.
		\end{obs*}

	It turns out that every extension arises in this way and we can determine which choices of $g$ give equivalent extensions. 

		\begin{theorem}[Ringel] 
			For $U,W \in Rep_{\mathbb{F}} Q$ define a map $$\gamma_{W,U}:\bigoplus_{i \in Q_{0}} Hom_{\mathbb{F}}(W_i,U_i) \rightarrow \bigoplus_{\alpha \in Q_{1}} Hom_{\mathbb{F}}(W_{s(\alpha)},U_{t(\alpha)}) $$ Via $$ \theta \mapsto (U_{\alpha} \circ \theta_{s(\alpha)} - \theta_{t(\alpha)} \circ W_{\alpha})_{\alpha \in Q_1}.$$ Then there are isomorphisms $ker(\gamma_{W,U}) \cong Hom_Q(W,U)$ and $coker(\gamma_{W,U}) \cong Ext_Q(W,U)$.
		\end{theorem}
		
		\begin{proof}
			The kernel of $\gamma_{W,U}$ is exactly the set of maps $\theta :W \rightarrow U$ with $U_{\alpha} \circ \theta_{s(\alpha)} = \theta_{t(\alpha)} \circ W_{\alpha}$, i.e. the set $Hom_Q(W,U)$. We already have the map $\bigoplus_{\alpha \in Q_{1}} Hom_{\mathbb{F}} (W_{s(\alpha)},U_{t(\alpha)}) \rightarrow Ext_Q(W,U)$. \\ Given any extension $0 \rightarrow U \rightarrow V \rightarrow W \rightarrow 0$, there is a splitting $V_i= W_i \oplus U_i$ of $\mathbb{F}$-vector spaces and since U is a subrepresentation of V under these splittings the map $V_{\alpha}$ takes the form $\begin{bmatrix}W_{\alpha} & 0\\ g_{\alpha} & U_{\alpha} \end{bmatrix}$ for some $g_{\alpha} \in Hom_{\mathbb{F}}(W_{s(\alpha)}, U_{t(\alpha)}).$ \\ It remains to show that the kernel of this map is the image of $\gamma_{W,U}$. Consider $g \in \bigoplus_{\alpha \in Q_1} Hom_{\mathbb{F}}(W_{s(\alpha)},U_{t(\alpha)})$ such that $[V_g]=[W\oplus U]$. That is we have a commutative diagram,
		\[ \begin{tikzcd}
			0 \arrow{r} & U \arrow[equal]{d} \arrow{r}& V_g \arrow{r}\arrow{d}{\theta} & W  \arrow[equal]{d} \arrow{r} & 0 \\
			0 \arrow{r} & U \arrow{r} & W\oplus U \arrow{r} & W \arrow{r}& 0 
			\end{tikzcd}\]
write $\theta_i =\begin{bmatrix} \theta_{11}^{(i)} & \theta_{12}^{(i)}\\ \theta_{21}^{(i)} &\theta_{22}^{(i)} \end{bmatrix}$ with $\theta_{11}^{(i)}:W_i \rightarrow W_i$, $\theta_{12}^{(i)}:U_i \rightarrow W_i$, $\theta_{21}^{(i)}:W_i \rightarrow U_i$, and $\theta_{22}^{(i)}:U_i \rightarrow U_i$. Commutativity of the squares immediately gives $\theta_{11}^{(i)}=id_{W_i}$, $\theta_{22}^{(i)}=id_{U_i}$, and $\theta_{12}^{(i)}=0$
		\end{proof}

		\begin{recall*}
			$\underline{dim}(U) = (dim(U))_{i \in Q_0} \in \mathbb{Z}^{Q_0}$ is the \underline{dimension vector} of U. 
		\end{recall*}
		\begin{definition}
			The \underline{Euler-Ringel form} $<.,.>:\mathbb{Z}^{Q_0} \times \mathbb{Z}^{Q_0} \rightarrow \mathbb{Z}$ is the bilinear form given by $$\displaystyle{<\underline{w},\underline{u}>:= \sum_{i \in Q_0} w_i u_i - \sum_{\alpha \in Q_1} w_{s(\alpha)} u_{t(\alpha)}}.$$
		\end{definition}

		\begin{corollary}
			For $U,W \in rep_\mathbb{F} Q$, we have $dim(Hom_Q(W,U))-dim(Ext_Q(W,U)) = <\underline{dim}W, \underline{dim}U>$.
We will use the notation $<W,U>$ to represent this.
		\end{corollary}

		\begin{example}
			$Q : 1 \leftarrow 2$ \\
			Q has irreducible representations:
			$$\mathbb{F} \leftarrow 0 =S_1=P_1$$
			$$0 \leftarrow \mathbb{F} =S_2=I_2$$
			$$\mathbb{F} \xleftarrow{id} \mathbb{F} = I_1 =P_2.$$
			With
			$$ 0 \rightarrow S_1 \rightarrow P_2 \rightarrow S_2 \rightarrow 0$$
			$$<S_2,S_1>=0-1=-1$$
			but $Hom_Q(S_2,S_1)=0$, so $Ext_Q(S_2,S_1) \cong \mathbb{F}$.
			We need to check $ 0 \rightarrow S_1 \rightarrow P_2 \rightarrow S_2 \rightarrow 0 $ is not the trivial extension.
			Does there exist a $\theta$ such that the following diagram commutes?
			\[ \begin{tikzcd}
			0 \arrow{r} & S_1 \arrow[equal]{d} \arrow{r}& P_2\arrow{r}\arrow{d}{\theta} & S_2  \arrow[equal]{d} \arrow{r} & 0 \\
			0 \arrow{r} & S_1 \arrow{r} & S_1 \oplus S_2 \arrow{r} & S_2 \arrow{r}& 0 
			\end{tikzcd}\]
			This would give a surjective map $P_2 \twoheadrightarrow S_1$, but no such map exists by the Diagram below.
			\[ \begin{tikzcd}
			\mathbb{F} \arrow[d, "\lambda \neq 0" ']&  \arrow{l}{id} \mathbb{F} \arrow{d}  \\
			\mathbb{F}  & \arrow{l} 0
			\end{tikzcd}\]
			So $[P_2]$ is nontrivial and thus spans $Ext_Q(S_2,S_1).$
		\end{example}

		\begin{example}
			$Q : 1 \stackbin[\beta]{\alpha}{\leftleftarrows} 2$ \\
			Q has simple representations:
			$$\mathbb{F} \leftleftarrows 0 =S_1$$
			$$0 \leftleftarrows \mathbb{F} =S_2$$
			
			With
			$$<S_1,S_2>=0$$
			$$<S_2,S_1>=0-2=-2.$$
			Since $dim(Hom_Q(S_2,S_1))=0$ we get $dim(Ext_Q(S_2,S_1)) = 2$.
			The indecomposables of Q with dimension vector $(1,1)$ are $$\delta_{(\lambda,\mu)} : \mathbb{F}  \stackbin[\mu]{\lambda}{\leftleftarrows} \mathbb{F}$$ where ${(\lambda,\mu)} \neq (0,0)$.
			Which of these $\delta_{(\lambda,\mu)}$ are isomorphic?\\
			Clearly $\delta_{(\lambda,\mu)} \cong \delta_{(k\lambda,k\mu)}$ for any nonzero scalar $k\in \mathbb{F}$
			\[ \begin{tikzcd}
			\mathbb{F} \arrow[d, "k \neq 0" '] &  \arrow[l, "\lambda"',yshift=0.7ex] \arrow[l,"\mu",yshift=-0.7ex]  \mathbb{F} \arrow{d}{1}  \\
			\mathbb{F}  &  \arrow[l, "k\lambda"',yshift=0.7ex] \arrow[l,"k\mu",yshift=-0.7ex]  \mathbb{F}
			\end{tikzcd}\]
		\end{example}

		\begin{prop}
			Indecomposable representations of $Q : 1 \stackbin[\beta]{\alpha}{\leftleftarrows} 2$ with dimension vector $(1,1)$ are parameterized by points of $\mathbb{P}^1$.
		\end{prop}
		\begin{proof}
			Consider two points $[\lambda:\mu] ,[\lambda ' : \mu '] \in \mathbb{P}^1$. Assume $\delta_{(\lambda,\mu)}$ and $\delta_{(\lambda ',\mu ')}$ 				are isomorphic.
			\[ \begin{tikzcd}
			\mathbb{F} \arrow[d, "k \neq 0" '] &  \arrow[l, "\lambda"',yshift=0.7ex] \arrow[l,"\mu",yshift=-0.7ex]  \mathbb{F} \arrow{d}{k ' \neq 0}  \\
			\mathbb{F}  &  \arrow[l, "\lambda ' "',yshift=0.7ex] \arrow[l,"\mu ' ",yshift=-0.7ex]  \mathbb{F}
			\end{tikzcd}\]
			This being a map of quiver representations implies that $k \lambda = \lambda ' k'$ and $k\mu =\mu ' k'$.
			If $\lambda = 0$ then $[\lambda:\mu] =[\lambda ' : \mu ']=[0:1]$, and if  $\mu = 0$ then $[\lambda:\mu] =[\lambda ' : \mu ']=[1:0]$.
			Otherwise we have $\frac{\lambda}{\mu} = \frac{\lambda'}{\mu'}$ again giving $[\lambda:\mu] =[\lambda ' : \mu ']$.
		\end{proof}
		So what are the natural generators of $Ext_Q(S_2,S_1)$?
		\begin{exercise}
			Show that $\lambda [\delta_{[1:0]}] + \mu [\delta_{[0:1]}] =[\delta_{[\lambda:\mu]}]  $ and thus $$ 0 \rightarrow S_1 \rightarrow \delta_{[1:0]} \rightarrow S_2 \rightarrow 0$$ and $$ 0 \rightarrow S_1 \rightarrow \delta_{[0:1]} \rightarrow S_2 \rightarrow 0$$ give a basis for $Ext_Q(S_2,S_1)$.
		\end{exercise}
		Now we can see that $<\delta,S_2>=1-0=1$ but $Hom_Q(\delta,S_2) \cong \mathbb{F}$ and so $Ext_Q(\delta,S_2)=0$.
		Similarly $<S_1,\delta>=1$ with $dim(Hom_Q(S_1,\delta)) =1$ and so $Ext_Q(S_1,\delta)=0$. Also $<\delta,S_1>=1-2=-1$ and $dim(Hom_Q(\delta,S_1)) =0$ implies  $dim(Ext_Q(\delta,S_1))=1$. Similarly $dim(Ext_Q(S_2,\delta))=1$.
		
  \section{Root Systems and Gabriel's Theorem}

  \section{Auslander-Reiten Theory}


%%%%%%%%%%%%%%%%%%%%%%%
\chapter{Hall Algebras}

  \section{Basic Definitions}

  \section{Bialgebra Structure}

  \section{Relation to Symmetric Functions}

  \section{Ringel's Theorem}


%%%%%%%%%%%%%%%%%%%%%%%%%%%%%%%%%%%%%%%%%%%%%%%%%%%%%%%%%%%
\chapter{Canonical Bases for Quantized Enveloping Algebras}

  \section{Lusztig Quiver Varieties}

  \section{IC Sheaves}

  \section{Crystal Bases}


%%%%%%%%%%%%%%%%%%%%%%%%%%%%%%%%%%%%%%%%%%%%%%%%%%%%%%%%%%%%%%%%
\chapter{Representation Theory of Quantized Enveloping Algebras}


%%%%%%%%%%%%%%%%%%%%%%
\chapter{Applications}

  \section{Integrable Lattice Models}

  \section{Knot Theory}



\end{document}
