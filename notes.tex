\documentclass{book}
\usepackage{amsmath,amssymb,latexsym}
\usepackage[margin=1in]{geometry}

\title{From Quivers to Quantum Groups}

\begin{document}
  \maketitle

%%%%%%%%%%%%%%%%%%%%%%
\chapter{Introduction}

  These are notes for a graduate course at University of Notre Dame, Spring 2018.


%%%%%%%%%%%%%%%%%%%%%%%%%%%%%%%%
\chapter{Quiver Representations}

  \section{Motivation: Classification Problems in Linear Algebra}

  \section{Basic Definitions}

  \section{Root Systems and Gabriel's Theorem}

  \section{Auslander-Reiten Theory}


%%%%%%%%%%%%%%%%%%%%%%%
\chapter{Hall Algebras}

  \section{Basic Definitions}

  \section{Bialgebra Structure}

  \section{Relation to Symmetric Functions}

  \section{Ringel's Theorem}


%%%%%%%%%%%%%%%%%%%%%%%%%%%%%%%%%%%%%%%%%%%%%%%%%%%%%%%%%%%
\chapter{Canonical Bases for Quantized Enveloping Algebras}

  \section{Lusztig Quiver Varieties}

  \section{IC Sheaves}

  \section{Crystal Bases}


%%%%%%%%%%%%%%%%%%%%%%%%%%%%%%%%%%%%%%%%%%%%%%%%%%%%%%%%%%%%%%%%
\chapter{Representation Theory of Quantized Enveloping Algebras}


%%%%%%%%%%%%%%%%%%%%%%
\chapter{Applications}

  \section{Integrable Lattice Models}

  \section{Knot Theory}



\end{document}
