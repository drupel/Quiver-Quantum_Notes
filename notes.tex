\documentclass{book}
\usepackage{amsmath,amssymb,latexsym,mathtools,tikz-cd}
\usepackage[margin=1in]{geometry}

\title{From Quivers to Quantum Groups}
\author{Dylan Rupel}

% Define theorem-like environments
 \newtheorem{theorem}{Theorem}[section]
 \newtheorem{lemma}[theorem]{Lemma}
 \newtheorem{corollary}[theorem]{Corollary}
 \newtheorem{prop}[theorem]{Proposition}
 \newtheorem{definition}[theorem]{Definition}
 \newtheorem{remark}[theorem]{Remark}
 \newtheorem{example}{Example}[section]
 \newtheorem {exercise}[theorem] {Exercise}
 % Define math operators
 \DeclareMathOperator{\rank}{rank}
 \DeclareMathOperator{\Hom}{Hom}
 \DeclareMathOperator{\Rep}{Rep}
 \DeclareMathOperator{\rep}{rep}
 \DeclareMathOperator{\im}{im}

\begin{document}
  \maketitle

%%%%%%%%%%%%%%%%%%%%%%
\chapter{Introduction}

  These are notes for a graduate course at University of Notre Dame, Spring 2018.

  \section{Acknowledgements}

%%%%%%%%%%%%%%%%%%%%%%%%%%%%%%%%
\chapter{Quiver Representations}

  \section{Motivation: Classification Problems in Linear Algebra}
  
  Study configurations of vector spaces and linear maps up to simultaneous base change.
  
  	\subsection{Baby Examples}
  		\begin{enumerate}
			\item Vector spaces: completely classified by dimension. $V \cong \mathbb{C}^n$ for some $n$.
			\item A single linear map: $f:U\to V$. There are three invariants: $\dim U, \dim V, \rank f$. Up to base change, $f$ can be represented uniquely in a left-justified reduced row-echelon form.
			\item Endomorphism: $f:V\to V$. Have to fix $\dim V=n$. How to classify $n\times n$ matrices up to conjugation? Use Jordan Canonical Form: $f$ is completely determined by its eigenvalues and sizes/multiplicities of its Jordan blocks. 
  		\end{enumerate}

  	\subsection{Subspace Problems}
		\begin{enumerate}
			\item One Subspace Problem: How to classify $U \subseteq V$? Fix $\dim U$ and $\dim V$. By changing bases in $V$, a subspace can be moved to any other subspace of the same dimension.
			\item Two Subspace Problem:  $U_1,U_2 \subseteq V$. Fix $\dim U_i, \dim V,$ and $\dim (U_1\cap U_2)$. This is enough to classify such configurations.
			\item Three Subspace Problem: $U_1,U_2,U_3 \subseteq V$. How to classify ordered triples of subspaces in a fixed vector space up to simultaneous base change?  Fix $\dim U_i, \dim V, \dim (U_i\cap U_j),$ and $\dim (U_1\cap U_2\cap U_3)$. This alone is not enough.
				\begin{example}
					Let $\dim V=3$ and $\dim U_i=1$. Consider \[V=\mathbb{C}^3, U_1 = \mathbb{C} \begin{bmatrix}1\\0\\0\end{bmatrix},U_2 = \mathbb{C} \begin{bmatrix}0\\1\\0\end{bmatrix},U_3 = \mathbb{C} \begin{bmatrix}0\\0\\1\end{bmatrix}\] and  \[V=\mathbb{C}^3, U_1 = \mathbb{C} \begin{bmatrix}1\\0\\0\end{bmatrix},U_2 = \mathbb{C} \begin{bmatrix}0\\1\\0\end{bmatrix},U_3 = \mathbb{C} \begin{bmatrix}1\\1\\0\end{bmatrix}.\] These are not equivalent.
				\end{example}
			So we must also fix $\dim ((U_i + U_j)\cap U_k)$. This turns out to be enough.
      			\item Four Subspace Problem: $U_1,U_2,U_3, U_4 \subseteq V$. Fix $\dim U_i, \dim V, \dim (U_i\cap U_j),\dim (U_i\cap U_j\cap U_k), \dim ((U_i + U_j)\cap U_k),\dim (U_1\cap U_2 \cap U_3 \cap U_4),  \dim ((U_i + U_j+U_k)\cap U_l),$ and $ \dim ((U_i+ U_j)\cap (U_k+U_l)).$ This is not enough.
				\begin{example}
					Let $\dim V=2$, and consider \[V=\mathbb{C}^2, U_1 = \mathbb{C} \begin{bmatrix}1\\0\end{bmatrix},U_2 = \mathbb{C} \begin{bmatrix}0\\1\end{bmatrix},U_3 = \mathbb{C} \begin{bmatrix}1\\1\end{bmatrix},U_4 = \mathbb{C} \begin{bmatrix}1\\ \lambda\end{bmatrix}\] where $\lambda\neq 0,1$. We need to know this continuous parameter $\lambda$ to distinguish such configurations. 
				\end{example}
			More generally: the endomorphism problem embeds into the four subspace problem. For $f:U\to U$, we consider \[V=U\oplus U, \ U_1=U\oplus 0, \ U_2=0\oplus U, \ U_3=\{(x,x):x\in U\},\ U_4=\{(x,f(x)):x\in U\}.\] A base change in $V$ fixing $U_1,U_2,U_3$ amounts to conjugation of $f$. 
			\item Five Subspace Problem: Hopeless...
		\end{enumerate}
		

  \section{Basic Definitions}
  
	\begin{definition} A \textbf{quiver} $Q=(Q_0,Q_1,s,t)$ consists of a set $Q_0$ of vertices, a set $Q_1$ of arrows, and maps $s,t:Q_1\to Q_0$. For $\alpha\in Q_1$, we write $\alpha:s(\alpha)\to t(\alpha).$
 	 \end{definition}
  
  	\begin{definition} Fix a field $\mathbb{F}$. An \textbf{$\mathbb{F}$-representation} of $Q$ is a set \[V= \left(\{V_i\}_{i\in Q_0},\{V_\alpha\}_{\alpha\in Q_1} \right)\] where $V_i$ is an $\mathbb{F}$-vector space and $V_\alpha : V_{s(\alpha)}\to V_{t(\alpha)}$ is an $\mathbb{F}$-linear map.
   	 \end{definition}
    
  	\begin{example}
    	Let $Q$ be the quiver $1\rightarrow 3\leftarrow 2$ with $Q_0=\{1,2,3\}$. A representation $V$ can be drawn as $V_1 \xrightarrow{f_1} V_3 \xleftarrow{f_2} V_2$. For $f_i$ injective, this gives an instance of the two subspace problem. Changing bases in $V_1,V_2,V_3$ amounts to conjugating each $f_i$ by automorphisms of the $V_i$, say $g_i:V_i\to V_i$. That is, the configuration $V_1 \xrightarrow{g_3 f_1 g_1^{-1}} V_3 \xleftarrow{g_3 f_2 g_2^{-1}} V_2$ is equivalent to the original. Putting all these together, 
		\[ \begin{tikzcd}
		V_1 \arrow{r}{f_1} \arrow{d}{g_1} & V_3 \arrow{d}{g_3}& V_2 \arrow{l}{f_2}\arrow{d}{g_2} \\
		V_1 \arrow{r}{g_3f_1g_1^{-1}}& V_3 & V_2 \arrow{l}{g_3f_2g_2^{-1}}
		\end{tikzcd}\]
	where each square commutes.
    	\end{example}
    
   	 \begin{definition}
    	A \textbf{morphism} $\theta:V\to W$ between representations $V$ and $W$ of $Q$ consists of $\mathbb{F}$-linear maps $\theta_i:V_i\to W_i$ such that $\theta_{t(\alpha)} \circ V_\alpha = W_\alpha\circ \theta_{s(\alpha)}$ for all $\alpha\in Q_1$. That is,
	\[ \begin{tikzcd}
		V_{s(\alpha)} \arrow{r}{V_\alpha} \arrow{d}{\theta_{s(\alpha)}} & V_{t(\alpha)} \arrow{d}{\theta_{t(\alpha)}}\\
		W_{s(\alpha)} \arrow{r}{W_\alpha}& W_{t(\alpha)}
		\end{tikzcd}\]
		commutes for every $\alpha\in Q_1$.
    	\end{definition}
    
   	 \begin{definition}
    	Let $\Rep_\mathbb{F}Q$ be the category of all $\mathbb{F}$-representations of $Q$. (Here, the identity maps of representations have all components the identity maps on the $V_i$.) Write $\rep_\mathbb{F}Q \subseteq \Rep_\mathbb{F}Q$ for the full subcategory consisting of representations with finite dimensional spaces $V_i$.
    	\end{definition}
    
    	\begin{lemma}
    		An isomorphism in $\Rep_\mathbb{F}Q$ is any map $\theta:V\to W$ with each $\theta_i:V_i\to W_i$ invertible.
    	\end{lemma}
    
    	Thus each classification problem is the study of isomorphism classes in $\Rep_\mathbb{F}Q$ for some quiver $Q$.

	\begin{definition}
		Given $v\in \rep_\mathbb{F}Q$, write $\underline{\dim} V = (\dim V_i)_{i\in Q_0}\in \mathbb{N}^{Q_0}$ for the \textbf{dimension vector} of $V$.
	\end{definition}
	
	Fix a dimension vector $\underline{d} \in \mathbb{N}^{Q_0}$. Consider the affine space $M_{\underline{d}}(Q)=\bigoplus_{\alpha\in Q_1} \Hom_{\mathbb{F}}\left(\mathbb{F}^{d_{s(\alpha)}},\mathbb{F}^{d_{t(\alpha)}}\right)$. This consists of all representations of $Q$ of a fixed dimension vector. The group $G_{\underline{d}}= \prod_{i\in Q_0} GL_{d_i}(\mathbb{F})$ acts on $M_{\underline{d}}(Q)$ by $g\cdot (f_\alpha)_{\alpha\in Q_1}=\left(g_{t(\alpha)}f_\alpha g_{s(\alpha)}^{-1}\right)_{\alpha\in Q_1}.$
	
	\begin{lemma}
		For any $g\in G_{\underline{d}}$ and $(f_\alpha)_{\alpha\in Q_1}$, the representations corresponding to $(f_\alpha)_{\alpha\in Q_1}$ and  $g\cdot (f_\alpha)_{\alpha\in Q_1}$ are isomorphic.
	\end{lemma}
	
	Thus the classification problem amounts to understanding the $G_{\underline{d}}$-orbit structure of $M_{\underline{d}}(Q)$.
	
	\subsection{Properties of $\Rep_\mathbb{F}Q$}
		$Rep_\mathbb{F}Q$ is an additive category: For $V,W\in \Rep_\mathbb{F}Q$, write $\Hom_Q(V,W)$ for the set of morphisms from $V$ to $W$. For $\theta,\theta' \in \Hom_Q(V,W)$ we have another morphism $\theta+\theta'$ with $(\theta+\theta')_i=\theta_i+\theta'_i$. This gives $\Hom_Q(V,W)$ the structure of an abelian group. 
		
		\begin{definition}
			Given $V,W\in \Rep_\mathbb{F}Q$ define $V\oplus W$ to be the representation with $(V\oplus W)_i=V_i\oplus W_i$ and $(V\oplus W)_\alpha = V_\alpha \oplus W_\alpha$. 
		\end{definition}
		
		\begin{definition}
			A representation $V \in \Rep_\mathbb{F}Q$ is \textbf{indecomposable} if it can't be written as $U\oplus W$ for any nontrivial representations $U,W$.
		\end{definition}
		
		\begin{example}
			Let $Q: \ 1\to3\leftarrow 2$. Let $V_1 \xrightarrow{f_1} V_3 \xleftarrow{f_2} V_2$ be a representation of $Q$. Considering $\ker f_1$ and $\ker f_2$, we get a direct sum decomposition where  $$\mathbb{F}^{\dim(\ker f_1)} \rightarrow 0 \leftarrow 0$$ and $$ 0\rightarrow 0 \leftarrow \mathbb{F}^{\dim(\ker f_2)}$$ split off. Thus we can assume $f_1$ and $f_2$ are both injective. Considering $\im f_1\cap \im f_2 \subseteq V_3$, we get a summand of the form \[(\mathbb{F}\rightarrow\mathbb{F}\leftarrow\mathbb{F})^{\dim(\im f_1\cap \im f_2)}.\] Then we can assume $\im f_1\cap \im f_2=0$. Then we get summands of the form 
			\begin{align*} V_1\rightarrow\im f_1 \leftarrow 0 &\cong (\mathbb{F}\rightarrow\mathbb{F}\leftarrow 0)^{\dim V_1} \\ 0\rightarrow\im f_2 \leftarrow V_2 &\cong (0 \rightarrow\mathbb{F}\leftarrow\mathbb{F})^{\dim V_2} \end{align*}
			Thus we may assume $V_1$ and $V_2$ are zero and we get a summand \[(0 \rightarrow\mathbb{F}\leftarrow0)^{\dim V_3}. \]
		\end{example}
		
		\begin{definition}
			Given $\theta:V\to W$, define $K=(K_i,K_\alpha)$ where $K_i=\ker\theta_i$ and \[ \begin{tikzcd}
		K_{s(\alpha)} \arrow[dashed]{r}{K_\alpha} \arrow{d} & K_{t(\alpha)} \arrow{d}\\
		V_{s(\alpha)} \arrow{r}{V_\alpha} \arrow{d}{\theta_{s(\alpha)}} & V_{t(\alpha)} \arrow{d}{\theta_{t(\alpha)}}\\
		W_{s(\alpha)} \arrow{r}{W_\alpha}& W_{t(\alpha)}
		\end{tikzcd}\]
		\end{definition}
		
		\begin{lemma}
			$K$ is the kernel of $\theta$ in the category $\Rep_\mathbb{F}Q$.
		\end{lemma}

  \section{Root Systems and Gabriel's Theorem}

  \section{Auslander-Reiten Theory}


%%%%%%%%%%%%%%%%%%%%%%%
\chapter{Hall Algebras}

  \section{Basic Definitions}

  \section{Bialgebra Structure}

  \section{Relation to Symmetric Functions}

  \section{Ringel's Theorem}


%%%%%%%%%%%%%%%%%%%%%%%%%%%%%%%%%%%%%%%%%%%%%%%%%%%%%%%%%%%
\chapter{Canonical Bases for Quantized Enveloping Algebras}

  \section{Lusztig Quiver Varieties}

  \section{IC Sheaves}

  \section{Crystal Bases}


%%%%%%%%%%%%%%%%%%%%%%%%%%%%%%%%%%%%%%%%%%%%%%%%%%%%%%%%%%%%%%%%
\chapter{Representation Theory of Quantized Enveloping Algebras}


%%%%%%%%%%%%%%%%%%%%%%
\chapter{Applications}

  \section{Integrable Lattice Models}

  \section{Knot Theory}



\end{document}
