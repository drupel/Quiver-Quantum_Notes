\documentclass{book}
\usepackage{amsmath,amssymb,latexsym,mathtools,tikz-cd,amsthm,stackrel}
\usepackage[margin=1in]{geometry}

\title{From Quivers to Quantum Groups}
\author{Dylan Rupel}

% Define theorem-like environments
 \newtheorem{theorem}{Theorem}[section]
 \newtheorem{lemma}[theorem]{Lemma}
 \newtheorem{corollary}[theorem]{Corollary}
 \newtheorem{prop}[theorem]{Proposition}
 \newtheorem{definition}[theorem]{Definition}
 \newtheorem{remark}[theorem]{Remark}
 \newtheorem{example}{Example}[section]
 \newtheorem {exercise}[theorem] {Exercise}
 \newtheorem{note}[theorem]{Note}
 \newtheorem{terminology}[theorem]{Terminology}
 \newtheorem* {obs*}{Observation}
 \newtheorem* {recall*}{Recall}
 \newtheorem{claim}[theorem]{Claim}
 \newtheorem* {facts*}{Facts}

% We can decide how to typeset terms when they are defined later. For now, put them in italics.
\newcommand{\defterm}{\emph}

% A flag to keep track of things that need work:
\newcommand{\problematic}[1]{\textcolor{red}{#1}}
 
 % Define math operators
 \DeclareMathOperator{\rank}{rank}
 \DeclareMathOperator{\Hom}{Hom}
 \DeclareMathOperator{\Rep}{Rep}
 \DeclareMathOperator{\rep}{rep}
 \DeclareMathOperator{\im}{im}
 \DeclareMathOperator{\coker}{coker}
 \newcommand{\id}{\mathrm{id}}
 \DeclareMathOperator{\rad}{rad}
 \DeclareMathOperator{\End}{End}
 \DeclareMathOperator{\Ext}{Ext}
 \DeclareMathOperator{\Aut}{Aut}

\begin{document}
  \maketitle

%%%%%%%%%%%%%%%%%%%%%%
\chapter{Introduction}

  These are notes for a graduate course at University of Notre Dame, Spring 2018.

  \section{Acknowledgements}
  Thanks to Kathryn Burton, Danny Orton, \dots for typing of the first rough draft of these notes.

%%%%%%%%%%%%%%%%%%%%%%%%%%%%%%%%
\chapter{Quiver Representations}

  \section{Motivation: Classification Problems in Linear Algebra}
  
  Study configurations of vector spaces and linear maps up to simultaneous base change.
  
  	\subsection{Baby Examples}
  		\begin{enumerate}
			\item Vector spaces: completely classified by dimension. $V \cong \mathbb{C}^n$ for some $n$.
			\item A single linear map: $f:U\to V$. There are three invariants: $\dim U, \dim V, \rank f$. Up to base change, $f$ can be represented uniquely in a left-justified reduced row-echelon form.
			\item Endomorphism: $f:V\to V$. Have to fix $\dim V=n$. How to classify $n\times n$ matrices up to conjugation? Use Jordan Canonical Form: $f$ is completely determined by its eigenvalues and sizes/multiplicities of its Jordan blocks. 
  		\end{enumerate}

  	\subsection{Subspace Problems}
		\begin{enumerate}
			\item One Subspace Problem: How to classify $U \subseteq V$? Fix $\dim U$ and $\dim V$. By changing bases in $V$, a subspace can be moved to any other subspace of the same dimension.
			\item Two Subspace Problem:  $U_1,U_2 \subseteq V$. Fix $\dim U_i, \dim V,$ and $\dim (U_1\cap U_2)$. This is enough to classify such configurations.
			\item Three Subspace Problem: $U_1,U_2,U_3 \subseteq V$. How to classify ordered triples of subspaces in a fixed vector space up to simultaneous base change?  Fix $\dim U_i, \dim V, \dim (U_i\cap U_j),$ and $\dim (U_1\cap U_2\cap U_3)$. This alone is not enough.
				\begin{example}
					Let $\dim V=3$ and $\dim U_i=1$. Consider \[V=\mathbb{C}^3, U_1 = \mathbb{C} \begin{bmatrix}1\\0\\0\end{bmatrix},U_2 = \mathbb{C} \begin{bmatrix}0\\1\\0\end{bmatrix},U_3 = \mathbb{C} \begin{bmatrix}0\\0\\1\end{bmatrix}\] and  \[V=\mathbb{C}^3, U_1 = \mathbb{C} \begin{bmatrix}1\\0\\0\end{bmatrix},U_2 = \mathbb{C} \begin{bmatrix}0\\1\\0\end{bmatrix},U_3 = \mathbb{C} \begin{bmatrix}1\\1\\0\end{bmatrix}.\] These are not equivalent.
				\end{example}
			So we must also fix $\dim ((U_i + U_j)\cap U_k)$. This turns out to be enough.
      			\item Four Subspace Problem: $U_1,U_2,U_3, U_4 \subseteq V$. Fix $\dim U_i, \dim V, \dim (U_i\cap U_j),\dim (U_i\cap U_j\cap U_k), \dim ((U_i + U_j)\cap U_k),\dim (U_1\cap U_2 \cap U_3 \cap U_4),  \dim ((U_i + U_j+U_k)\cap U_l),$ and $ \dim ((U_i+ U_j)\cap (U_k+U_l)).$ This is not enough.
				\begin{example}
					Let $\dim V=2$, and consider \[V=\mathbb{C}^2, U_1 = \mathbb{C} \begin{bmatrix}1\\0\end{bmatrix},U_2 = \mathbb{C} \begin{bmatrix}0\\1\end{bmatrix},U_3 = \mathbb{C} \begin{bmatrix}1\\1\end{bmatrix},U_4 = \mathbb{C} \begin{bmatrix}1\\ \lambda\end{bmatrix}\] where $\lambda\neq 0,1$. We need to know this continuous parameter $\lambda$ to distinguish such configurations. 
				\end{example}
			More generally: the endomorphism problem embeds into the four subspace problem. For $f:U\to U$, we consider \[V=U\oplus U, \ U_1=U\oplus 0, \ U_2=0\oplus U, \ U_3=\{(x,x):x\in U\},\ U_4=\{(x,f(x)):x\in U\}.\] A base change in $V$ fixing $U_1,U_2,U_3$ amounts to conjugation of $f$. 
			\item Five Subspace Problem: Hopeless...
		\end{enumerate}
		

  \section{Basic Definitions}
  
	\begin{definition} A \textbf{quiver} $Q=(Q_0,Q_1,s,t)$ consists of a set $Q_0$ of vertices, a set $Q_1$ of arrows, and maps $s,t:Q_1\to Q_0$. For $\alpha\in Q_1$, we write $\alpha:s(\alpha)\to t(\alpha).$
 	 \end{definition}
  
  	\begin{definition} Fix a field $\mathbb{F}$. An \textbf{$\mathbb{F}$-representation} of $Q$ is a set \[V= \left(\{V_i\}_{i\in Q_0},\{V_\alpha\}_{\alpha\in Q_1} \right)\] where $V_i$ is an $\mathbb{F}$-vector space and $V_\alpha : V_{s(\alpha)}\to V_{t(\alpha)}$ is an $\mathbb{F}$-linear map.
   	 \end{definition}
    
  	\begin{example}
    	Let $Q$ be the quiver $1\rightarrow 3\leftarrow 2$ with $Q_0=\{1,2,3\}$. A representation $V$ can be drawn as $V_1 \xrightarrow{f_1} V_3 \xleftarrow{f_2} V_2$. For $f_i$ injective, this gives an instance of the two subspace problem. Changing bases in $V_1,V_2,V_3$ amounts to conjugating each $f_i$ by automorphisms of the $V_i$, say $g_i:V_i\to V_i$. That is, the configuration $V_1 \xrightarrow{g_3 f_1 g_1^{-1}} V_3 \xleftarrow{g_3 f_2 g_2^{-1}} V_2$ is equivalent to the original. Putting all these together, 
		\[ \begin{tikzcd}
		V_1 \arrow{r}{f_1} \arrow{d}{g_1} & V_3 \arrow{d}{g_3}& V_2 \arrow{l}{f_2}\arrow{d}{g_2} \\
		V_1 \arrow{r}{g_3f_1g_1^{-1}}& V_3 & V_2 \arrow{l}{g_3f_2g_2^{-1}}
		\end{tikzcd}\]
	where each square commutes.
    	\end{example}
    
   	 \begin{definition}
    	A \textbf{morphism} $\theta:V\to W$ between representations $V$ and $W$ of $Q$ consists of $\mathbb{F}$-linear maps $\theta_i:V_i\to W_i$ such that $\theta_{t(\alpha)} \circ V_\alpha = W_\alpha\circ \theta_{s(\alpha)}$ for all $\alpha\in Q_1$. That is,
	\[ \begin{tikzcd}
		V_{s(\alpha)} \arrow{r}{V_\alpha} \arrow{d}{\theta_{s(\alpha)}} & V_{t(\alpha)} \arrow{d}{\theta_{t(\alpha)}}\\
		W_{s(\alpha)} \arrow{r}{W_\alpha}& W_{t(\alpha)}
		\end{tikzcd}\]
		commutes for every $\alpha\in Q_1$.
    	\end{definition}
    
   	 \begin{definition}
    	Let $\Rep_\mathbb{F}Q$ be the category of all $\mathbb{F}$-representations of $Q$. (Here, the identity maps of representations have all components the identity maps on the $V_i$.) Write $\rep_\mathbb{F}Q \subseteq \Rep_\mathbb{F}Q$ for the full subcategory consisting of representations with finite dimensional spaces $V_i$.
    	\end{definition}
    
    	\begin{lemma}
    		An isomorphism in $\Rep_\mathbb{F}Q$ is any map $\theta:V\to W$ with each $\theta_i:V_i\to W_i$ invertible.
    	\end{lemma}
    
    	Thus each classification problem is the study of isomorphism classes in $\Rep_\mathbb{F}Q$ for some quiver $Q$.

	\begin{definition}
		Given $v\in \rep_\mathbb{F}Q$, write $\underline{\dim} V = (\dim V_i)_{i\in Q_0}\in \mathbb{N}^{Q_0}$ for the \textbf{dimension vector} of $V$.
	\end{definition}
	
	Fix a dimension vector $\underline{d} \in \mathbb{N}^{Q_0}$. Consider the affine space $M_{\underline{d}}(Q)=\bigoplus_{\alpha\in Q_1} \Hom_{\mathbb{F}}\left(\mathbb{F}^{d_{s(\alpha)}},\mathbb{F}^{d_{t(\alpha)}}\right)$. This consists of all representations of $Q$ of a fixed dimension vector. The group $G_{\underline{d}}= \prod_{i\in Q_0} GL_{d_i}(\mathbb{F})$ acts on $M_{\underline{d}}(Q)$ by $g\cdot (f_\alpha)_{\alpha\in Q_1}=\left(g_{t(\alpha)}f_\alpha g_{s(\alpha)}^{-1}\right)_{\alpha\in Q_1}.$
	
	\begin{lemma}
		For any $g\in G_{\underline{d}}$ and $(f_\alpha)_{\alpha\in Q_1}$, the representations corresponding to $(f_\alpha)_{\alpha\in Q_1}$ and  $g\cdot (f_\alpha)_{\alpha\in Q_1}$ are isomorphic.
	\end{lemma}
	
	Thus the classification problem amounts to understanding the $G_{\underline{d}}$-orbit structure of $M_{\underline{d}}(Q)$.
	
	\subsection{Properties of $\Rep_\mathbb{F}Q$}
		$Rep_\mathbb{F}Q$ is an additive category: For $V,W\in \Rep_\mathbb{F}Q$, write $\Hom_Q(V,W)$ for the set of morphisms from $V$ to $W$. For $\theta,\theta' \in \Hom_Q(V,W)$ we have another morphism $\theta+\theta'$ with $(\theta+\theta')_i=\theta_i+\theta'_i$. This gives $\Hom_Q(V,W)$ the structure of an abelian group. 
		
		\begin{definition}
			Given $V,W\in \Rep_\mathbb{F}Q$ define $V\oplus W$ to be the representation with $(V\oplus W)_i=V_i\oplus W_i$ and $(V\oplus W)_\alpha = V_\alpha \oplus W_\alpha$. 
		\end{definition}
		
		\begin{definition}
			A representation $V \in \Rep_\mathbb{F}Q$ is \textbf{indecomposable} if it can't be written as $U\oplus W$ for any nontrivial representations $U,W$.
		\end{definition}
		
		\begin{example}
			Let $Q: \ 1\to3\leftarrow 2$. Let $V_1 \xrightarrow{f_1} V_3 \xleftarrow{f_2} V_2$ be a representation of $Q$. Considering $\ker f_1$ and $\ker f_2$, we get a direct sum decomposition where  $$\mathbb{F}^{\dim(\ker f_1)} \rightarrow 0 \leftarrow 0$$ and $$ 0\rightarrow 0 \leftarrow \mathbb{F}^{\dim(\ker f_2)}$$ split off. Thus we can assume $f_1$ and $f_2$ are both injective. Considering $\im f_1\cap \im f_2 \subseteq V_3$, we get a summand of the form \[(\mathbb{F}\rightarrow\mathbb{F}\leftarrow\mathbb{F})^{\dim(\im f_1\cap \im f_2)}.\] Then we can assume $\im f_1\cap \im f_2=0$. Then we get summands of the form 
			\begin{align*} V_1\rightarrow\im f_1 \leftarrow 0 &\cong (\mathbb{F}\rightarrow\mathbb{F}\leftarrow 0)^{\dim V_1} \\ 0\rightarrow\im f_2 \leftarrow V_2 &\cong (0 \rightarrow\mathbb{F}\leftarrow\mathbb{F})^{\dim V_2} \end{align*}
			Thus we may assume $V_1$ and $V_2$ are zero and we get a summand \[(0 \rightarrow\mathbb{F}\leftarrow0)^{\dim V_3}. \]
		\end{example}
		
		\begin{definition}
			Given $\theta:V\to W$, define $K=(K_i,K_\alpha)$ where $K_i=\ker\theta_i$ and \[ \begin{tikzcd}
		K_{s(\alpha)} \arrow[dashed]{r}{K_\alpha} \arrow{d} & K_{t(\alpha)} \arrow{d}\\
		V_{s(\alpha)} \arrow{r}{V_\alpha} \arrow{d}{\theta_{s(\alpha)}} & V_{t(\alpha)} \arrow{d}{\theta_{t(\alpha)}}\\
		W_{s(\alpha)} \arrow{r}{W_\alpha}& W_{t(\alpha)}
		\end{tikzcd}\]
		\end{definition}
		
		
		\begin{lemma}
			The representation $K$ is the kernel of $\theta:V\to W$ in the category $\Rep_\mathbb{F}Q$.
		\end{lemma}
		
		\begin{proof}
		    Suppose $\varphi:x\to V$ is a morphism of representations satisfying $\theta\circ\varphi=0$. We need to show there exists a unique map $\psi:X\to K$ so that the diagram 
		    \[ \begin{tikzcd}
		    & X \arrow[dashed]{dl} \arrow{d}{\varphi}&\\
		    K\arrow[hookrightarrow]{r} & V \arrow{r}{\theta} & W 
		    \end{tikzcd}\]
		    commutes. For any $x_i\in X_i$, we have $\theta_i(\varphi_i(x_i))=0$. So $\varphi_i(x_i)\in\ker\theta_i$ and we must define $\psi_i(x_i)=\varphi_i(x_i)$. To see that these give a morphism of representations, we check for $\alpha\in Q_1$ and $x_{s(\alpha)}\in X_{s(\alpha)}$:
		    \[K_\alpha\circ\psi_{s(\alpha)}(x_{s(\alpha)})=V_\alpha\circ\varphi_{s(\alpha)}(x_{s(\alpha)})=\varphi_{t(\alpha)}\circ X_\alpha(x_{s(\alpha)})=\psi_{t(\alpha)}\circ X_\alpha(x_{s(\alpha)}).\]
		\end{proof}
		
        \begin{definition}
            Given $\theta:V\to W$, define a representation $C$ with $C_i=\coker\theta_i$ for $i\in Q_0$ and $C_\alpha$ for $\alpha\in Q_1$ as the map induced by $W_\alpha$ in this diagram:
            \[ \begin{tikzcd}
	    	V_{s(\alpha)} \arrow{r}{V_\alpha} \arrow{d}{\theta_{s(\alpha)}} & V_{t(\alpha)} \arrow{d}{\theta_{t(\alpha)}}\\
	    	W_{s(\alpha)} \arrow{d} \arrow{r}{W_\alpha}& W_{t(\alpha)}\arrow{d}\\
	    	C_{s(\alpha)} \arrow[dashed]{r}{C_\alpha}  & C_{t(\alpha)}
	    	\end{tikzcd}\]
        \end{definition}
        
        \begin{exercise}
            Prove that $C$ is the cokernel of the morphism $\theta:V\to W$ in the category $\Rep_\mathbb{F}Q$. \\ Hint: Show that dualizing all vector spaces and linear maps gives a contravariant functor $\rep_\mathbb{F}Q \to \rep_\mathbb{F}Q^{op}$ which takes cokernels in $\rep_\mathbb{F}Q$ to kernels in $\rep_\mathbb{F}Q^{op}$.
        \end{exercise}

        \begin{definition}
            A representation $U$ is called a \textbf{subrepresentation} of a representation $V$ if there exists a monomorphism of representations $\theta: U\to V$, i.e. each $\theta_i$ is an injective linear map. Define $V/U$ as the cokernel of $\theta$ in this case.
        \end{definition}
        
        \begin{theorem}[First Isomorphism Theorem]
            For any morphism $\theta: V\to W$ of representations, we have an isomorphism $\im\theta\cong V / \ker\theta$.
        \end{theorem}
        
        \begin{proof}
            Observe that $\im\theta=(\theta_i(v_i),I_\alpha)$ where for $v_{s(\alpha)}\in V_{s(\alpha)}$ we get \[I_\alpha(\theta_{s(\alpha)}(v_{s(\alpha)}))=\theta_{t(\alpha)}(V_\alpha(v_{s(\alpha)}))\] and $V/\ker\theta=(V_i+\ker\theta_i,Q_\alpha)$ where \[Q_\alpha(v_{s(\alpha)}+\ker\theta_{s(\alpha)})=V_\alpha(v_{s(\alpha)})+\ker\theta_{t(\alpha)}.\] But each $\theta_i$ is $\mathbb{F}$-linear and so there are isomorphisms 
                \begin{align*}
                    \bar{\theta}_i:V_i/\ker\theta_i &\to \im\theta_i \\
                    v_i+\ker\theta_i &\mapsto \theta_i(v_i).
                \end{align*}
            These satisfy $\bar{\theta}_{t(\alpha)}\circ Q_\alpha=I_\alpha\circ\bar{\theta}_{s(\alpha)}$ and so we get an isomorphism $V/\ker\theta\cong\im\theta$.
        \end{proof}
        
        \begin{theorem}
            Let $\mathbb{F}$ be a field and $Q$ a quiver. With the above notions of kernel and cokernel, $\Rep_\mathbb{F}Q$ has the structure of an $\mathbb{F}$-linear abelian category.
        \end{theorem}
        
        \begin{proof}
            Last time we observed $\mathbb{F}$-linearity and additive structure. 
                \begin{exercise}
                    Check that $\Hom()\times\Hom()\to\Hom()$ is bilinear.
                \end{exercise}
            It remains to observe that given a morphism $\theta:V\to W$ with kernel $\iota:K\to V$ and cokernel $\pi:W\to C$, we have $\coker\iota=\ker\theta.$ But this is essentially the statement of the First Isomorphism Theorem.
        \end{proof}
        
        \begin{example}
            We saw last time that the two subspace quiver $1\to 3 \leftarrow 2$ has six isomorphism classes of indecomposables:
                \begin{align*}
                    S_1 &= \mathbb{F} \to 0 \leftarrow 0 \\
                    S_3 &= 0 \to \mathbb{F} \leftarrow 0 \\
                    S_2 &= 0 \to 0 \leftarrow \mathbb{F} \\
                    P_1 &= \mathbb{F} \xrightarrow{\id} \mathbb{F} \leftarrow 0 \\
                    P_2 &= 0 \to \mathbb{F} \xleftarrow{\id} \mathbb{F} \\
                    I_3 &= \mathbb{F} \xrightarrow{\id} \mathbb{F} \xleftarrow{\id} \mathbb{F}
                \end{align*}
            Consider $x\in \Hom_Q(S_3,I_3).$ 
                \[ \begin{tikzcd}
		            0 \arrow{r}{0} \arrow{d}{0} & \mathbb{F} \arrow{d}{x\neq 0} & 0 \arrow{l}{0}\arrow{d}{0}\\
		            \mathbb{F} \arrow{r} \arrow{d} & \mathbb{F}\arrow{d} & \arrow{l} \arrow{d}\mathbb{F} \\
		            \mathbb{F} \arrow{r} & 0 & \mathbb{F} \arrow{l}
		        \end{tikzcd}\]
		    So $\Hom_Q(S_3,I_3)=\mathbb{F}$ and $\coker(x)\cong S_1\oplus S_2.$ In a similar way, we determine that $\Hom_Q(I_3,P_1)=0$ using the diagram below. 
		        \[ \begin{tikzcd}
		            \mathbb{F} \arrow{r} \arrow{d}{0} & \mathbb{F} \arrow{d}{0} & \mathbb{F} \arrow{l}{\id}\arrow{d}{0}\\
		            \mathbb{F} \arrow{r} & \mathbb{F} & 0 \arrow{l}
		        \end{tikzcd}\]
		    Considering all possible homomorphisms between indecomposable representations, we can construct what is called the Auslander-Reiten Quiver for our example:
		        \[\begin{tikzcd}
		            & P_1 \arrow{dr} & & S_2 \\
		            S_3 \arrow{ur}\arrow{dr} && I_3 \arrow{ur}\arrow{dr} \\
		            & P_2 \arrow{ur} && S_1
		        \end{tikzcd}\]
        \end{example}




	\subsection{Extensions}
		\begin{definition} 
			Let U,V, and W be representations of a quiver Q. We call V an \underline{extension of W by U} if there exists an exact sequence $0 \rightarrow U \xrightarrow{\phi} V 				\xrightarrow{\psi} W \rightarrow 0$. \\ i.e. $\phi$ is injective, $\psi$ is surjective, and $\im \phi = \ker \psi$.
		\end{definition}
	We call two extensions $V$ and $V'$ of $W$ by $U$ equivalent if there exists an isomorphism $\theta : V \rightarrow V'$ making the following diagram commute.
		\[ \begin{tikzcd}
		0 \arrow{r} & U \arrow[equal]{d} \arrow{r}{\phi}& V \arrow{r}{\psi}\arrow{d}{\theta} & W  \arrow[equal]{d} \arrow{r} & 0 \\
		0 \arrow{r} & U \arrow{r}{\phi '} & V' \arrow{r}{\psi '} & W \arrow{r}& 0 
		\end{tikzcd}\]
	This clearly defines an equivalence relation on the set of extensions of $W$ by $U$, and we write $ Ext_Q(W,U)$ for the set of equivalences classes of these extensions. 

		\begin{exercise}
			Consider a commutative diagram 
			\[ \begin{tikzcd}
			0 \arrow{r} & U \arrow{d}{\phi} \arrow{r}& V \arrow{r}\arrow{d} & W  \arrow{d}{\psi} \arrow{r} & 0 \\
			0 \arrow{r} & U' \arrow{r} & V' \arrow{r} & W' \arrow{r}& 0 
			\end{tikzcd}\]
			Show that the right hand square is a pullback if and only if $\phi$ is an isomorphism, and show that the left hand square is a pushout if and only if $\psi$ is an isomorphism.
		\end{exercise}
		\begin{definition} 
			Given $[V],[V'] \in Ext_Q(W,U)$, define their \underline{Baer sum} $[V]+[V']=[V'']$ by the following commutative diagram:
			\[ \begin{tikzcd}
			0 \arrow{r} & U \oplus U \arrow[equal]{d} \arrow{r}& V \oplus V' \arrow{r} & W \oplus W \arrow{r}  \arrow[dl, phantom, "\urcorner", very near start] & 0 \\
			0 \arrow{r} & U \oplus U  \arrow[dr, phantom, "\ulcorner", very near start] \arrow[d, "(1 \text{,}-1)"'] \arrow{r} & X \arrow{r}\arrow{u} \arrow{d} & W \arrow[u, "\Delta"'] \arrow{r}\arrow[equal]{d}& 0 \\
			0 \arrow{r} & U \arrow{r} & V'' \arrow{r} & W \arrow{r}& 0 
			\end{tikzcd}\]
		\end{definition}
	It is immediate from the universal property of the pullback that equivalent choices of $V$ and $V'$ produce equivalent $X's$ and then the universal property of the pushout shows that the class of $V''$ does not depend on the choice of representatives for $[V]$ and $[V']$. 

		\begin{prop}
			The Baer sum defines an abelian group structure on the set $Ext_Q(W,U)$ for any $U,W \in Rep_\mathbb{F}Q$ where $[W\oplus U]$ is the identity element and given $[V] \in Ext_Q(W,U)$ represented by a sequence $0 \rightarrow U \xrightarrow{\phi} V \xrightarrow{\psi} W \rightarrow 0$, the class $-[V]$ is represented by the sequence $0 \rightarrow U \xrightarrow{-\phi} V \xrightarrow{\psi} W \rightarrow 0$ or equivalently $0 \rightarrow U \xrightarrow{\phi} V \xrightarrow{-\psi} W \rightarrow 0$

		\end{prop}

		\begin{obs*}
			Let $U$ and $W$ be representations of $Q$. For any choice of maps $g_{\alpha} :W_{s(\alpha)} \rightarrow U_{t(\alpha)}$ for $\alpha \in Q_1$, we can build an extension $V_g$ of $W$ by $U$ where $V_i = W_i \oplus U_i$ for $i \in Q_0$ and $V_{\alpha} = \begin{bmatrix}W_{\alpha} & 0\\ g_{\alpha} & U_{\alpha} \end{bmatrix}$ for $\alpha \in Q_1$.
		\end{obs*}

	It turns out that every extension arises in this way and we can determine which choices of $g$ give equivalent extensions. 

		\begin{theorem}[Ringel] 
			For $U,W \in Rep_{\mathbb{F}} Q$ define a map $$\gamma_{W,U}:\bigoplus_{i \in Q_{0}} Hom_{\mathbb{F}}(W_i,U_i) \rightarrow \bigoplus_{\alpha \in Q_{1}} Hom_{\mathbb{F}}(W_{s(\alpha)},U_{t(\alpha)}) $$ Via $$ \theta \mapsto (U_{\alpha} \circ \theta_{s(\alpha)} - \theta_{t(\alpha)} \circ W_{\alpha})_{\alpha \in Q_1}.$$ Then there are isomorphisms $ker(\gamma_{W,U}) \cong Hom_Q(W,U)$ and $coker(\gamma_{W,U}) \cong Ext_Q(W,U)$.
		\end{theorem}
		
		\begin{proof}
			The kernel of $\gamma_{W,U}$ is exactly the set of maps $\theta :W \rightarrow U$ with $U_{\alpha} \circ \theta_{s(\alpha)} = \theta_{t(\alpha)} \circ W_{\alpha}$, i.e. the set $Hom_Q(W,U)$. We already have the map $\bigoplus_{\alpha \in Q_{1}} Hom_{\mathbb{F}} (W_{s(\alpha)},U_{t(\alpha)}) \rightarrow Ext_Q(W,U)$. \\ Given any extension $0 \rightarrow U \rightarrow V \rightarrow W \rightarrow 0$, there is a splitting $V_i= W_i \oplus U_i$ of $\mathbb{F}$-vector spaces and since U is a subrepresentation of V under these splittings the map $V_{\alpha}$ takes the form $\begin{bmatrix}W_{\alpha} & 0\\ g_{\alpha} & U_{\alpha} \end{bmatrix}$ for some $g_{\alpha} \in Hom_{\mathbb{F}}(W_{s(\alpha)}, U_{t(\alpha)}).$ \\ It remains to show that the kernel of this map is the image of $\gamma_{W,U}$. Consider $g \in \bigoplus_{\alpha \in Q_1} Hom_{\mathbb{F}}(W_{s(\alpha)},U_{t(\alpha)})$ such that $[V_g]=[W\oplus U]$. That is we have a commutative diagram,
		\[ \begin{tikzcd}
			0 \arrow{r} & U \arrow[equal]{d} \arrow{r}& V_g \arrow{r}\arrow{d}{\theta} & W  \arrow[equal]{d} \arrow{r} & 0 \\
			0 \arrow{r} & U \arrow{r} & W\oplus U \arrow{r} & W \arrow{r}& 0 
			\end{tikzcd}\]
write $\theta_i =\begin{bmatrix} \theta_{11}^{(i)} & \theta_{12}^{(i)}\\ \theta_{21}^{(i)} &\theta_{22}^{(i)} \end{bmatrix}$ with $\theta_{11}^{(i)}:W_i \rightarrow W_i$, $\theta_{12}^{(i)}:U_i \rightarrow W_i$, $\theta_{21}^{(i)}:W_i \rightarrow U_i$, and $\theta_{22}^{(i)}:U_i \rightarrow U_i$. Commutativity of the squares immediately gives $\theta_{11}^{(i)}=id_{W_i}$, $\theta_{22}^{(i)}=id_{U_i}$, and $\theta_{12}^{(i)}=0$
		\end{proof}

		\begin{recall*}
			$\underline{dim}(U) = (dim(U))_{i \in Q_0} \in \mathbb{Z}^{Q_0}$ is the \underline{dimension vector} of U. 
		\end{recall*}
		\begin{definition}
			The \underline{Euler-Ringel form} $<.,.>:\mathbb{Z}^{Q_0} \times \mathbb{Z}^{Q_0} \rightarrow \mathbb{Z}$ is the bilinear form given by $$\displaystyle{<\underline{w},\underline{u}>:= \sum_{i \in Q_0} w_i u_i - \sum_{\alpha \in Q_1} w_{s(\alpha)} u_{t(\alpha)}}.$$
		\end{definition}

		\begin{corollary}
			For $U,W \in rep_\mathbb{F} Q$, we have $dim(Hom_Q(W,U))-dim(Ext_Q(W,U)) = <\underline{dim}W, \underline{dim}U>$.
We will use the notation $<W,U>$ to represent this.
		\end{corollary}

		\begin{example}
			$Q : 1 \leftarrow 2$ \\
			Q has irreducible representations:
			$$\mathbb{F} \leftarrow 0 =S_1=P_1$$
			$$0 \leftarrow \mathbb{F} =S_2=I_2$$
			$$\mathbb{F} \xleftarrow{id} \mathbb{F} = I_1 =P_2.$$
			With
			$$ 0 \rightarrow S_1 \rightarrow P_2 \rightarrow S_2 \rightarrow 0$$
			$$<S_2,S_1>=0-1=-1$$
			but $Hom_Q(S_2,S_1)=0$, so $Ext_Q(S_2,S_1) \cong \mathbb{F}$.
			We need to check $ 0 \rightarrow S_1 \rightarrow P_2 \rightarrow S_2 \rightarrow 0 $ is not the trivial extension.
			Does there exist a $\theta$ such that the following diagram commutes?
			\[ \begin{tikzcd}
			0 \arrow{r} & S_1 \arrow[equal]{d} \arrow{r}& P_2\arrow{r}\arrow{d}{\theta} & S_2  \arrow[equal]{d} \arrow{r} & 0 \\
			0 \arrow{r} & S_1 \arrow{r} & S_1 \oplus S_2 \arrow{r} & S_2 \arrow{r}& 0 
			\end{tikzcd}\]
			This would give a surjective map $P_2 \twoheadrightarrow S_1$, but no such map exists by the Diagram below.
			\[ \begin{tikzcd}
			\mathbb{F} \arrow[d, "\lambda \neq 0" ']&  \arrow{l}{id} \mathbb{F} \arrow{d}  \\
			\mathbb{F}  & \arrow{l} 0
			\end{tikzcd}\]
			So $[P_2]$ is nontrivial and thus spans $Ext_Q(S_2,S_1).$
		\end{example}

		\begin{example}
			$Q : 1 \stackbin[\beta]{\alpha}{\leftleftarrows} 2$ \\
			Q has simple representations:
			$$\mathbb{F} \leftleftarrows 0 =S_1$$
			$$0 \leftleftarrows \mathbb{F} =S_2$$
			
			With
			$$<S_1,S_2>=0$$
			$$<S_2,S_1>=0-2=-2.$$
			Since $dim(Hom_Q(S_2,S_1))=0$ we get $dim(Ext_Q(S_2,S_1)) = 2$.
			The indecomposables of Q with dimension vector $(1,1)$ are $$\delta_{(\lambda,\mu)} : \mathbb{F}  \stackbin[\mu]{\lambda}{\leftleftarrows} \mathbb{F}$$ where ${(\lambda,\mu)} \neq (0,0)$.
			Which of these $\delta_{(\lambda,\mu)}$ are isomorphic?\\
			Clearly $\delta_{(\lambda,\mu)} \cong \delta_{(k\lambda,k\mu)}$ for any nonzero scalar $k\in \mathbb{F}$
			\[ \begin{tikzcd}
			\mathbb{F} \arrow[d, "k \neq 0" '] &  \arrow[l, "\lambda"',yshift=0.7ex] \arrow[l,"\mu",yshift=-0.7ex]  \mathbb{F} \arrow{d}{1}  \\
			\mathbb{F}  &  \arrow[l, "k\lambda"',yshift=0.7ex] \arrow[l,"k\mu",yshift=-0.7ex]  \mathbb{F}
			\end{tikzcd}\]
		\end{example}

		\begin{prop}
			Indecomposable representations of $Q : 1 \stackbin[\beta]{\alpha}{\leftleftarrows} 2$ with dimension vector $(1,1)$ are parameterized by points of $\mathbb{P}^1$.
		\end{prop}
		\begin{proof}
			Consider two points $[\lambda:\mu] ,[\lambda ' : \mu '] \in \mathbb{P}^1$. Assume $\delta_{(\lambda,\mu)}$ and $\delta_{(\lambda ',\mu ')}$ 				are isomorphic.
			\[ \begin{tikzcd}
			\mathbb{F} \arrow[d, "k \neq 0" '] &  \arrow[l, "\lambda"',yshift=0.7ex] \arrow[l,"\mu",yshift=-0.7ex]  \mathbb{F} \arrow{d}{k ' \neq 0}  \\
			\mathbb{F}  &  \arrow[l, "\lambda ' "',yshift=0.7ex] \arrow[l,"\mu ' ",yshift=-0.7ex]  \mathbb{F}
			\end{tikzcd}\]
			This being a map of quiver representations implies that $k \lambda = \lambda ' k'$ and $k\mu =\mu ' k'$.
			If $\lambda = 0$ then $[\lambda:\mu] =[\lambda ' : \mu ']=[0:1]$, and if  $\mu = 0$ then $[\lambda:\mu] =[\lambda ' : \mu ']=[1:0]$.
			Otherwise we have $\frac{\lambda}{\mu} = \frac{\lambda'}{\mu'}$ again giving $[\lambda:\mu] =[\lambda ' : \mu ']$.
		\end{proof}
		So what are the natural generators of $Ext_Q(S_2,S_1)$?
		\begin{exercise}
			Show that $\lambda [\delta_{[1:0]}] + \mu [\delta_{[0:1]}] =[\delta_{[\lambda:\mu]}]  $ and thus $$ 0 \rightarrow S_1 \rightarrow \delta_{[1:0]} \rightarrow S_2 \rightarrow 0$$ and $$ 0 \rightarrow S_1 \rightarrow \delta_{[0:1]} \rightarrow S_2 \rightarrow 0$$ give a basis for $Ext_Q(S_2,S_1)$.
		\end{exercise}
		Now we can see that $<\delta,S_2>=1-0=1$ but $Hom_Q(\delta,S_2) \cong \mathbb{F}$ and so $Ext_Q(\delta,S_2)=0$.
		Similarly $<S_1,\delta>=1$ with $dim(Hom_Q(S_1,\delta)) =1$ and so $Ext_Q(S_1,\delta)=0$. Also $<\delta,S_1>=1-2=-1$ and $dim(Hom_Q(\delta,S_1)) =0$ implies  $dim(Ext_Q(\delta,S_1))=1$. Similarly $dim(Ext_Q(S_2,\delta))=1$.

		\begin{example}
			Write $K(n)$ for the \underline{n-Kronecker quiver} $ \begin{tikzcd}
			1  &   \arrow[l, "\alpha_1"',yshift=1.5ex]  \arrow[l, phantom, "{\vdots}"',yshift=.5ex]     \arrow[l,"\alpha_n",yshift=-1.5ex] 2
			\end{tikzcd} $.\\ 
			Our goal is to classify indecomposable representations of $K(n)$ of dimension vector$(m,1)$.
			\[ \begin{tikzcd}
			\mathbb{F}^m  &   \arrow[l, "\alpha_1"',yshift=1.8ex]  \arrow[l, phantom, "{\vdots}"',yshift=0.8ex]     \arrow[l,"\alpha_n",yshift=-1.2ex] \mathbb{F}
			\end{tikzcd} \]
		\end{example}
		
		\begin{claim}		
			There are no indecomposables with dimension vector $(m,1)$ for $m>n$.
		\end{claim}
		
		\begin{proof}
			Given a representation $V\in M_{(m,1)}(K(n))$ the total image $\displaystyle \sum_{i=1}^n im(V_{\alpha_i})$ is at most $n$ dimensional. So for $m>n$ there are at least $m-n$ summands of $V$ isomorphic to $S_1$.
		\end{proof}

		\begin{claim}		
			There is a unique (up to isomorphism) indecomposable representation of dimension vector $(n,1)$.
		\end{claim}
		
		\begin{proof}
			Let $V, V' \in M_{(n,1)}(K(n))$ be indecomposable representations. We build an isomorphism $V \xrightarrow{\theta} V'$ as follows,
			\[ \begin{tikzcd}
			\mathbb{F}^m \arrow[d, "\theta_1" '] &  \arrow[l, "V_{\alpha_1}"',yshift=1.8ex]  \arrow[l, phantom, "{\vdots}"',yshift=0.8ex]     \arrow[l,"V_{\alpha_n}",yshift=-1.2ex] 				\mathbb{F} \arrow{d}{id=\theta_2}  \\ 
			\mathbb{F}^m  &  \arrow[l, "V'_{\alpha_1}"',yshift=1.5ex]  \arrow[l, phantom, "{\vdots}"',yshift=0.5ex]     \arrow[l,"V'_{\alpha_n}",yshift=-1.5ex] \mathbb{F}
			\end{tikzcd}\]
			by observing that $\{ V_{\alpha_i}(1)\}^n_{i=1}$ and $\{ V'_{\alpha_i}(1)\}^n_{i=1} $ are bases of $\mathbb{F}^n$ and so $\theta_1$ is the change of basis matrix giving isomorphism $\theta$.

		\end{proof}
	
		Write $P_2= \begin{tikzcd}
			\mathbb{F}^n  &   \arrow[l, "e_1"',yshift=1.5ex]  \arrow[l, phantom, "{\vdots}"',yshift=.5ex]     \arrow[l,"e_n",yshift=-1.5ex] \mathbb{F}
			\end{tikzcd}$, where $P_{2,\alpha_i}$ is the map picking out the $i^{th}$ standard basis vector. 

		\begin{claim}		
			For any indecomposable $V \in M_{(m,1)}(K(n))$ for $m\leq n$, we have $Hom_{K(n)}(P_2,V) \cong \mathbb{F}$.
		\end{claim}
		
		\begin{proof}
			\[ \begin{tikzcd}
			\mathbb{F}^n \arrow[d, "{[V_{\alpha_1}(1), \cdots ,V_{\alpha_n}(1)]}"' ] &  \arrow[l, yshift=1.8ex]  \arrow[l, phantom, "{\vdots}"',yshift=0.8ex]     							\arrow[l,yshift=-1.2ex] \mathbb{F} \arrow{d}{id} & = P_2  \\ 
			\mathbb{F}^m  &  \arrow[l, yshift=1.5ex]  \arrow[l, phantom, "{\vdots}"',yshift=0.5ex]     \arrow[l, yshift=-1.5ex] \mathbb{F} & = V
			\end{tikzcd}\]

		\end{proof}

		\begin{claim}		
			The indecomposable representations of dimension vector $(m,1)$ for $0\leq m \leq n$ are parameterized by $Gr_m(\mathbb{F}^n)$
		\end{claim}
		
		\begin{proof}
			They are parameterized by $m\times n$ matrices of rank $m$ up to $GL_m$ action on the left. 
		\end{proof}


		\subsection{Another Interpretation for Ext}

			Fix $\underline{d} \in \mathbb{N}^{Q_0}$, write $M_{\underline{d}}(Q) = \displaystyle \prod_{\alpha \in Q_1} Hom(\mathbb{F}^{d_{s(\alpha)}}, \mathbb{F}^{d_{t(\alpha)}})$ for the moduli representations of Q with dimension vector $\underline{d}$. $GL_{\underline{d}} =  \displaystyle \prod_{i \in Q_0} GL_{d_i}(\mathbb{F})$ acts on $M_{\underline{d}}(Q)$ by $$(g \cdot V)_{\alpha} = g_{t(\alpha)} V_{\alpha} g_{s(\alpha)}^{-1}.$$  This action is algebraic, $GL_{\underline{d}} \times M_{\underline{d}}(Q) \rightarrow M_{\underline{d}}(Q) .$ For $V \in M_{\underline{d}}(Q)$, write $\mathcal{O}_V := GL_{\underline{d}}\cdot V$ for the orbit through V. This is the set of all representations isomorphic to V. 

		\begin{facts*}
			
		\begin{enumerate}
		
			\item   $\mathcal{O}_V$ is a smooth subvariety of  $M_{\underline{d}}(Q)$.		
		  \item  $\mathcal{O}_V$ is open in $\overline{\mathcal{O}_V}$
	
		\end{enumerate}
		\end{facts*}

		\begin{definition}
			The \underline{Tits form} $q:\mathbb{N}^{Q_0} \rightarrow \mathbb{Z}$ is the quadratic form $q(\underline{d})=\left<\underline{d},\underline{d}\right> = \displaystyle \sum_{i \in Q_0} d_i^2 - \sum_{\alpha \in Q_1} d_{s(\alpha)}d_{t(\alpha)}.$
		\end{definition}

		\begin{lemma}
			$dim(M_{\underline{d}}(Q)) - dim(\mathcal{O}_V) = dim(End_Q(V))-q(\underline{d}) = dim(Ext_Q(V,V)).$
		\end{lemma}
		
		\begin{proof}
			$$dim(M_{\underline{d}}(Q)) =  \displaystyle \sum_{i \in Q_0} d_i^2 - q(\underline{d}) .$$

			\begin{eqnarray*}
				dim(\mathcal{O}_V) &=&  dim(GL_{\underline{d}}) - dim(Stab_{GL_{\underline{d}}}(V)) \\
							&=&   dim(GL_{\underline{d}}) - dim(Aut_{Q}(V)) \\
							&=&  \displaystyle \sum_{i \in Q_0} d_i^2  - dim(Aut_{Q}(V)) \\
							&=&  \displaystyle \sum_{i \in Q_0} d_i^2  - dim(End_{Q}(V)). 
			\end{eqnarray*}
			$$q(\underline{d}) = dim(Hom_Q(V,V)) - dim(Ext_Q(V,V)).$$
			The result follows.
		\end{proof}

		Let $N_{M_{\underline{d}}(Q)}(\mathcal{O}_V) = TM_{\underline{d}}(Q)|_{\mathcal{O}_V} /T\mathcal{O}_V$. The dimension of a fiber of $N_{M_{\underline{d}}(Q)}(\mathcal{O}_V)$ is  $dim(M_{\underline{d}}(Q)) - dim (\mathcal{O}_V) = dim Ext_Q(V,V)$. So the number $dim Ext_Q(V,V)$ is measuring the possible deformations of V. 

		\begin{definition}
			$V$ is \underline{rigid} if  $dim Ext_Q(V,V)=0.$
		\end{definition} 
		
		\begin{corollary}
Let $V$ be a representation of a quiver $Q$ and let $\underline{d}\in \mathbb{Z}^{Q_0}$ be a dimension vector.
\begin{enumerate}
\item If $\underline{d}\neq 0$ and $q(\underline{d})\leq 0$, then there are infinitely many orbits in $M_{\underline{d}}(Q)$.

\item The orbit $\mathcal{O}_V$ is open in $M_{\underline{d}}(Q)$ if and only if $V$ is rigid.

\item There is at most one rigid representation (up to isomorphism) of any given dimension vector.
\end{enumerate}
\end{corollary}

\begin{proof}
\begin{enumerate}
\item For $V\in M_{\underline{d}}(Q)$ nonzero, we have that $\textnormal{End}_Q(V)\neq 0$. Since $q(\underline{d})\leq 0$, the previous Lemma implies that $\dim \mathcal{O}_V< \dim M_{\underline{d}}(Q)$.

\item If $\mathcal{O}_V$ is open, then $\dim \mathcal{O}_V= \dim M_{\underline{d}}(Q)$ since $M_{\underline{d}}(Q)$ is irreducible. Conversely, if $V$ is rigid, we have $\overline{\mathcal{O}_V}= M_{\underline{d}}(Q)$. Since $\mathcal{O}_V$ is locally closed, conclude that it is open.

\item Suppose that $V,V'\in M_{\underline{d}}(Q)$ are both rigid. Irreducibility of $M_{\underline{d}}(Q)$ implies that $\mathcal{O}_V\cap \mathcal{O}_{V'}\neq \emptyset$. Therefore, $V\cong V'$, as needed.
\end{enumerate}
\end{proof}

\subsection{Path Algebras}

\begin{definition}
Let $Q=(Q_0,Q_1,s,t)$ be a (finite) quiver. Define the \underline{path algebra} $A=A_Q$ to be the $\mathbb{F}$-vector space spanned by all \underline{paths} in $Q$. A \underline{path} is a (possibly empty) sequence of arrows
$$
p=\alpha_n\alpha_{n-1}\cdots \alpha_1
$$
with $t(\alpha_i)=s(\alpha_{i+1})$ for $1\leq i \leq n-1$, or $e_i$ the lazy path at vertex $i$. We write $s(p)=s(\alpha_1)$ and $t(p)=t(\alpha_n)$. The multiplication $pq$ in $A$ of two paths $p$ and $q$ is given by concatenation if $s(p)=t(q)$ and is zero otherwise.
\end{definition}

\begin{remark}
If $e_i$ and $e_j$ are the lazy paths at vertices $i$ and $j$ respectively, then their product in $A$ is given by $e_ie_j=\delta_{i,j}e_i$, where $\delta_{i,j}$ denotes the Kronecker delta symbol.
\end{remark}

Write $A_0=\oplus_{i\in Q_0}\mathbb{F}e_i$ for the semi-simple algebra spanned by the orthogonal idempotents $e_i$. Write $A_1=\oplus_{\alpha\in Q_1}\mathbb{F}\alpha$ for the ($A_0$, $A_0$)-bimodule spanned by the arrows of $Q$ where
$$
e_i\alpha e_j= \delta_{i,t(\alpha)}\delta_{j,s(\alpha)}\alpha.
$$
For $n\geq 2$, set $A_n$ to be the $n$-fold tensor product of $A_1$ over $A_0$.

\begin{exercise}
Show that $A$ is isomorphic to the tensor algebra
$$
T_{A_0}(A_1)=\bigoplus_{n\geq 0} A_n.
$$
\end{exercise}

Our next goal will be to understand the projective $A$-modules. Below is a list of facts that will be useful:

\begin{enumerate}
\item $1_A=\sum_{i\in Q_0} e_i$

\item $Ae_i$ has basis of paths starting at vertex $i$

\item $e_jA$ has basis of paths ending at vertex $j$

\item $e_jAe_i$ has basis of paths starting at $i$ and ending at $j$

\item $Ae_iA$ has basis of paths passing through vertex $i$
\end{enumerate}

Therefore,
$$
A=\bigoplus_{i\in Q_0} Ae_i.
$$
It follows that each $Ae_i$ is a projective $A$-module. 
\begin{itemize}
\item For a left $A$-module $M$, we have $\Hom_A(Ae_i,M)\cong e_iM$, where the isomorphism is given by sending a morphism $f$ to $f(e_i)=f(e_i^2)=e_if(e_i)$.

\item For $x\in Ae_i$ and $y\in e_iA$ both nonzero, we have $xy\neq 0$. Indeed, for $p$ the longest path in $x$ and $q$ the longest path in $y$, the coefficient of $pq$ in $xy$ will be nonzero.

\item The $e_i$ are primitive idempotents. In particular, the modules $Ae_i$ are indecomposable. 

\begin{proof}
By the first bullet, $\textnormal{End}_A(Ae_i)\cong e_iAe_i$. If it contains an idempotent $f$, then we have $f^2=f=fe_i$, and so $f(f-e_i)=0$. By the second bullet, conclude that $f-e_i=0$, and so $Ae_i$ is indecomposable.
\end{proof}

\item If $e_i\in Ae_jA$, then $i=j$. Indeed, $Ae_jA$ are paths passing through vertex $j$.

\item The $e_i$ are \underline{inequivalent} idempotents, i.e. $Ae_i\neq Ae_j$ for all $i\neq j$. 

\begin{proof}
Suppose that $Ae_i\cong Ae_j$. By the first bullet, inverse isomorphisms correspond to elements $f\in e_iAe_j$ and $g\in e_jAe_i$. Then $fg=e_i$ and $gf=e_j$, so by the previous statement we must have $i=j$.
\end{proof}

\end{itemize}

\begin{corollary}
The $Ae_i=:P_i$ are exactly the collection of all non-isomorphic projective modules.
\end{corollary}

\begin{theorem}
For any $A$-module $M$, there is a canonical two-term projective resolution:
$$
0\longrightarrow A_{+}\otimes_{A_0}M\overset{\delta_M}\longrightarrow A\otimes_{A_0}M\overset{\epsilon_M}\longrightarrow M\longrightarrow 0
$$
where
\begin{itemize}
\item $A_{+}\otimes_{A_0}M=A\otimes_{A_0}A_1\otimes_{A_0}M=\bigoplus_{\alpha\in Q_1} A\otimes_{A_0}\mathbb{F}\alpha\otimes_{A_0}M=\bigoplus_{\alpha\in Q_1} Ae_{t(\alpha)}\otimes_{\mathbb{F}} e_{s(\alpha)}M$

\item $A\otimes_{A_0}M=\left( \bigoplus_{i\in Q_0} Ae_i\right)\otimes_{A_0}M=\bigoplus_{i\in Q_0}Ae_i\otimes_{A_0}e_iM=\bigoplus_{i\in Q_0}Ae_i\otimes_{\mathbb{F}}e_iM$

\item $\epsilon_M(x\otimes m)=xm$

\item $\delta_M(x\otimes y \otimes m)=xy\otimes m - x\otimes (ym)$, viewing $\delta_M$ as a map from $A\otimes_{A_0}A_1\otimes_{A_0}M$.
\end{itemize}
\end{theorem}

\begin{proof}
We need to check that this is an exact complex. It is clear that $\epsilon_M$ is surjective, and clear that $\im \delta_M \subseteq \ker \epsilon_M$

To see that $\delta_M$ is injective, suppose $z=\sum_i x_i\otimes y_i \otimes m_i$ is in the kernel of $\delta_M$. Consider $x_i$ which lives in the highest degree component. We get that $x_iy_i$ is in the highest degree component of $\delta_M(z)$, which implies that $x_iy_i=0$. Thus, $x_i=0$ or $y_i=0$, as needed.

We leave it as an exercise to show that $\ker \epsilon_M \subseteq \im \delta_M$.
\end{proof}

\begin{corollary}
The category $\textnormal{Mod}_{\mathbb{F}}A$ is \underline{hereditary}. In other words, it satisfies all of the following equivalent characterizations.
\begin{enumerate}
\item $A$ has global dimension one

\item For all $A$-modules $M,N$, we have $\textnormal{Ext}^i(M,N)=0$ for $i\geq 2$

\item Every nontrivial submodule of a projective module is projective
\end{enumerate}
\end{corollary}

\begin{theorem}
The categories $\textnormal{Rep}_{\mathbb{F}}Q$ and $\textnormal{Mod}_{\mathbb{F}}A_Q$ are equivalent.
\end{theorem}

\begin{proof}
Let $V$ be a representation of $Q$. Define an $A$-module 
$$
M_V=\bigoplus_{i\in Q_0} V_i
$$
with $A_1\otimes_{A_0}M_V\to M_V$ given by $\alpha\otimes v_{s(\alpha)}\mapsto V_{\alpha}(v_{s(\alpha)})$ for $v_{s(\alpha)}\in V_{s(\alpha)}$. In the other direction, given an $A$-module $M$, define a representation $V_M=(V_i,V_M)$ where $V_i=e_iM$ and $V_{\alpha}:V_{s(\alpha)}\to V_{t(\alpha)}$ is defined by $e_{s(\alpha)}m\mapsto e_{t(\alpha)}(\alpha m)$.
\end{proof}

\begin{exercise}
Check that these are inverse operations.
\end{exercise}

\subsection{Representation Type of Quivers}

\begin{definition}
A quiver $Q$ has \underline{finite representation type} if $\textnormal{Rep}_{\mathbb{F}}Q$ has finitely many indecomposable representations.
\end{definition}

Recall for a quiver $Q$, the Euler-Ringel form
$$
\langle -, - \rangle: \mathbb{Z}^{Q_0}\times \mathbb{Z}^{Q_0}\to \mathbb{Z}
$$
is given by $\langle \underline{d}, \underline{e}\rangle=\sum_{i\in Q_0} d_ie_i - \sum_{\alpha\in Q_1} d_{s(\alpha)}e_{t(\alpha)}$. It has associated quadratic form $q:\mathbb{Z}^{Q_0}\to \mathbb{Z}$ given by $q(\underline{d})=\langle \underline{d},\underline{d}\rangle$, and symmetric bilinear form $(\underline{d},\underline{e})=\langle \underline{d}, \underline{e}\rangle+\langle \underline{e}, \underline{d}\rangle$. Notice that $q(\underline{d})=(\underline{d},\underline{d})/2$ and $(\underline{d},\underline{e})=q(\underline{d}+\underline{e})-q(\underline{d})-q(\underline{e})$.


\begin{definition}
The quadratic form $q$ is \underline{positive definite} if $q(\underline{d})>0$ for all $\underline{d}\in\mathbb{Z}^{Q_0}$. $q$ is \underline{positive semidefinite} if $q(\underline{d})\geq 0$ for all $\underline{d}\in \mathbb{Z}^{Q_0}$. The \underline{radical} of $q$, written $\textnormal{rad}(q)$, is the set $\{ \underline{d}\in \mathbb{Z}^{Q_0} \mid (\underline{d},-)\equiv 0 \}$. For $\underline{d},\underline{e}\in \mathbb{Z}^{Q_0}$, write $\underline{d}\leq \underline{e}$ if $\underline{e}-\underline{z}\in \mathbb{N}^{Q_0}$. A vector $\underline{d}$ is \underline{sincere} if it has all nonzero entries.
\end{definition}

\begin{lemma}
Suppose $Q$ is connected and $\underline{d}\geq 0$ is a nonzero vector in the radical of $q$. Then the following hold:
\begin{enumerate}
\item $\underline{d}$ is sincere

\item $q$ is positive semidefinite

\item for any $\underline{e}\in \mathbb{Z}^{Q_0}$, the following are equivalent:
$$
q(\underline{e})=0 \iff \underline{e}\in \mathbb{Q}\underline{d} \iff \underline{e}\in \textnormal{rad}(q)
$$
\end{enumerate}
\end{lemma}

\begin{proof}
\begin{enumerate}
\item For any $i,j \in Q_0$, write $n_{i,j}$ for the number of arrows joining vertices $i$ and $j$. Then
$$
0=(\epsilon_i, \underline{d})=(2-2n_{i,i})d_i-\sum_{i\neq j} n_{i,j}d_j
$$
where $\epsilon_i$ is the dimension vector of the simple $s_i$. If $d_i=0$, then $\sum_{i\neq j}n_{i,j}d_j=0$, but each $d_j\geq 0$ and so we have $d_j=0$ for any vertex $j$ joined to $i$ by an edge because $Q$ is connected. Since $Q$ is connected, this implies that all entries of $\underline{d}$ are zero, a contradiction.

\item For $\underline{e}\in \mathbb{Z}^{Q_0}$, we have (taking $Q_0=\{1, \cdots ,n \}$)
\begin{align*}
0 & \leq \sum_{i<j} n_{i,j} \frac{d_id_j}{2}\left(\frac{e_i}{d_i}-\frac{e_j}{d_j}\right)^2\\
& = \sum_{i<j} n_{i,j}\frac{d_j}{2d_i}e_i^2-\sum_{i<j}n_{i,j}e_ie_j+\sum_{i<j}n_{i,j}\frac{d_i}{2d_j}e_j^2\\
& = \sum_{i\neq j} n_{i,j} \frac{d_j}{2d_i}e_i^2-\sum_{i<j}n_{i,j}e_ie_j\\
& = \sum_{i\in Q_0} \sum_{j \mid j\neq i} n_{i,j}\frac{d_j}{2d_i}e_i^2-\sum_{i<j}n_{i,j}e_ie_j\\
& = \sum_{i\in Q_0}(2-2n_{i,i})\frac{d_i}{2d_i}e_i^2-\sum_{i<j}n_{i,j}e_ie_j\\
& = \sum_{i\in Q_0}(1-n_{i,i})e_i^2-\sum_{i<j}n_{i.j}e_ie_j\\
& = \langle \underline{e}, \underline{e}\rangle = q(\underline{e})
\end{align*}

\item If $q(\underline{e})=0$, then by the first inequality above, we have $e_i/d_i=e_j/d_j$ whenever there is an edge joining $i$ and $j$. Since $Q$ is connected, this implies that $\underline{e}\in \mathbb{Q}\underline{d}$. Then $\underline{e}\in \textnormal{red}(q)$ by linearity of $q$.
\end{enumerate}
\end{proof}

\begin{definition}
A quiver $Q$ is \underline{Euclidean} if it is an orientation of one of the following graphs:
\end{definition}

\begin{lemma}
The quadratic form $q$ associated to any Euclidean quiver is positive semidefinite with radical vector $\delta>0$ as indicated in the above diagrams.
\end{lemma}

\begin{proof}
Exercise. Check directly that the given vectors are in the radical and apply the previous lemma.
\end{proof}

\begin{definition}
A quiver $Q$ is \underline{Dynkin} if it is an orientation of a Dynkin diagram of type $A$, $D$, or $E$.
\end{definition}

\begin{corollary}
The quadratic form $q$ associated to a Dynkin quiver is positive definite.
\end{corollary}

\begin{proof}
Embed the Dynkin quiver $Q$ into its corresponding Euclidean quiver $\tilde{Q}$. Then we get an embedding $\mathbb{Z}^{Q_0}\to \mathbb{Z}^{\tilde{Q}_0}$ which is compatible with the quadratic forms $q, \tilde{q}$. Since the image of this embedding consists of non-sincere vectors, we have that none of them are radical vectors and so $q(\underline{d})=\tilde{q}(\underline{d})>0$.
\end{proof}



  \section{Root Systems and Gabriel's Theorem}
  
\begin{theorem}
Let $Q$ be a connected quiver.
\begin{enumerate}
	\item If $Q$ is Euclidean, then the associated quadratic form $q$ is positive semi-definite with $\rad(q) = \mathbb{Z} \delta$.
	\item If $Q$ is Dynkin, then $q$ is positive definite.
	\item Otherwise, there exists $\underline d \geq 0$ with $q(\underline d) < 0$ and $(\underline d, \varepsilon_i) \leq 0$ for all $i$.
\end{enumerate}
\end{theorem}
\begin{proof}
We did $(1)$ and $(2)$ last time. For $(3)$, observe that if $Q$ is not Euclidean or Dynkin, then there is a subquiver (not necessarily full) $Q'$ of $Q$ which is Euclidean. Indeed, one of the following holds:
\begin{enumerate}
	\item $Q$ has the same vertex set as a Euclidean quiver with some extra arrows.
	\item $Q$ has a pair of vertices with more than two arrows -- then there is $\tilde A_2$ inside by removing arrows.
	\item There exists a pair of vertices with exactly two arrows between them by removing vertices.
	\item If $Q$ has unoriented cycles, then there is a $\tilde A_n$ inside.
	\item If $Q$ has no parallel arrows and no unoriented cycles:
	\begin{enumerate}
		\item $Q$ has a vertex of valence $\geq 4$ -- then there is $\tilde D_4$ inside.
		\item $Q$ has at least two vertices of valence 3 then some $\tilde D_n$ is inside.
		\item $Q$ has a single vertex of valence 3. Then there is an $\tilde E_6$, $\tilde E_7$, or $\tilde E_8$ inside.
	\end{enumerate}
\end{enumerate}

% Alternative argument:
% \begin{enumerate}
% 	\item If $Q$ has an unoriented cycle, then there is a $\tilde A_n$ inside.
% 	\item Otherwise, $Q$ is a tree.
% 	\begin{enumerate}
% 		\item If there are no branches, then $Q$ is an $A_n$.
% 		\item If there is a vertex of valence $\geq 4$, there is a $\tilde D_4$ inside.
% 		\item If there are two vertices of valence \geq 3, there is a $\tilde D_n$ inside.
% 		\item If there is exactly one vertex of valence $\geq 3$, then either we have $D_4, D_5$, or else there is a $\tilde E_6$, $\tilde E_7$, or $\tilde E_8$ inside.
% 	\end{enumerate}
% \end{enumerate}

Now take $\underline d = \delta$ for $Q'$. If $Q_0' = Q_0$, then $q(\underline d) < q'(\underline d) = 0$. Otherwise there exists $i \in Q_0 \setminus Q_0'$ and we can take $\underline d = 2\delta + \varepsilon_j$ for some fixed $j$ outside $Q_0'$ but connected to $Q'$. Then $q(\underline d) = \langle \underline d, \underline d \rangle = 4 \langle \delta, \delta \rangle + 2 \langle \delta, \varepsilon_j \rangle + 2 \langle \varepsilon_j, \delta \rangle + \langle \varepsilon_j, \varepsilon_j \rangle$. The sum of them cross terms is $\leq -2$ because $i$ lies outside of $Q_0'$. We have $\langle \varepsilon_j, \varepsilon_j \rangle \leq 1$, and we have $4 \langle \delta, \delta \rangle \leq 4\langle \delta, \delta \rangle' \leq 0$. Altogether, the sum is $\leq -1 < 0$.

Let us check that $(\underline d, \varepsilon_i) \leq 0$ for all $i$. We expand $(\underline d, \varepsilon_i) = (2\delta,\varepsilon_i) + (\varepsilon_j,\varepsilon_i)$. If $i \not \in Q_0'$ and $i \neq j$, then both terms are $\leq 0$. If $i=j \not \in Q_0'$, then the first term is $\leq -2$ (because $i$ is connected to $Q_0'$ and the second term is $\leq 2$. If $i \in Q_0'$ (so $i\neq j$), then the second term is zero, while the first term is $\leq 2(\delta, \varepsilon_i)_{Q_0'}$, which we can explicitly calculate to be 0.
\end{proof}

From now on, suppose that $Q$ is Euclidean or Dynkin.

\begin{definition}
Write $\Phi = \{\underline d \neq 0 \in \mathbb{Z}^{Q_0} \mid q(\underline d) \leq 1\}$ for the set of \defterm{roots}. We call $\underline d \in \Phi$ \defterm{real} if $q(\underline d) = 1$ and \defterm{imaginary} if $q(\underline d) = 0$.
\end{definition}

\begin{lemma}
\begin{enumerate}
	\item Each standard basis vector $\varepsilon_i$ is a real root.
	\item If $\underline d \in \Phi \cup \{0\}$, then $-\underline d$ and $\underline d \pm \underline e$ are also in $\Phi$ for any $e \in \rad(q)$. 
	\item For any $\underline d \in \Phi$, we have either $\underline d \geq 0$ or $\underline d \leq 0$.
	\item The set of imaginary roots is $\begin{cases} \emptyset & Q \text{ is Dynkin} \\ \mathbb{Z}\delta \setminus \{0\} & Q \text{ is Euclidean} \end{cases}$.
	\item If $Q$ is Euclidean, then $\Phi \cup \{0\} / \mathbb Z \delta$ is finite.
	\item If $Q$ is Dynkin, then $\Phi$ is finite.
\end{enumerate}
\end{lemma}

\begin{proof}
For $(2)$, first note that a quadratic form is invariant under multiplication by -1. Then we have $q(\underline d \pm \underline e) = q(\underline d) + q(\underline e) \pm (\underline d, \underline e) = \underline q(\underline d)$.

For $(3)$, write $\underline d = \underline d_+ - \underline d_-$ with $\underline d_+, \underline d_- \geq 0$, so $\underline d_+$ and $\underline d_-$ have disjoint support; suppose they are both nonzero. Then $(\underline d_+, \underline d_-) \leq 0$ because of the disjoint support. Thus $1 \geq q(\underline d) = q(\underline d_+) + q(\underline d_-)$. Because $q$ is positive semidefinite and integral, either $q(\underline d_+) = 0$ or $q(\underline d_-) = 0$, i.e. one of them is radical. Since it is nonzero, it is sincere, a contradiction since it vanishes on the support of $\underline d_-$.

For $(4)$, the lemma from last time shows that if $\rad(q) \neq 0$, then $Q$ is Euclidean and $\rad(q) = \mathbb{Z} \delta$.

For $(5)$, the key point is that $\{\underline d \in \Phi \cup \{0\} \mid d_i = 0 \text{ for some } i \text{ such that } \delta_i = 1\} \subseteq \{ \underline d \in \mathbb Z^n \mid - \delta \leq \underline d \leq \delta \}$.

For $(6)$, embed the set of roots into the set of roots of the Euclidean extension and apply $(5)$.
\end{proof}

\begin{example}
Let $Q$ be the Kronecker quiver $1 {}^\leftarrow_\leftarrow 2$. Then $\delta = (1,1)$  
\end{example}

\begin{definition}
$V \in \Rep_\mathbb{F} Q$ is a \defterm{brick} if $\End_Q(V) = \mathbb{F}$
\end{definition}

\begin{note}
 A brick is always an indecomposable.
\end{note}

\begin{lemma}
For indecomposable representations $V,W \in \Rep_\mathbb{F}(Q)$ with $\Ext_Q(V,W) = 0$, any nonzero map $\theta: V \to W$ is either injective or surjective.
\end{lemma}

\begin{remark}
This idea is foundational to a lot of things in tilting theory.
\end{remark} 

\begin{proof}
The map $\theta : V \to W$ gives rise to short exact sequences

\begin{equation}\label{eqn:eqn1}
0 \to \ker \theta \to V \to \im \theta \to 0
\end{equation}

\begin{equation}
0 \to \im \theta \to W \to \coker \theta \to 0
\end{equation}

Apply $\Hom(\coker \theta, -)$ to $(\ast)$. We get a long exact sequence whose tail is an epimorphism $\Ext(\coker \theta, V) \twoheadrightarrow \Ext(\coker \theta, \im \theta)$. So there is a commutative diagram

\begin{tikzcd}
0 \ar[r] & V \ar[r] \ar[d] & X \ar[r] \ar[d] & \coker \theta \ar[r] \ar[d] & 0 \\
0 \ar[r] & \im \theta \ar[r] & W \ar[r] & \coker \theta \ar[r] & 0
\end{tikzcd}

Since the left hand square is a pushout,  we get a short exact sequence

\begin{equation*}
0 \to V \to X \oplus \im \theta \to W \to 0
\end{equation*}

This is an extension of $W$ by $V$. By assumption, it splits. Since $V$ and $W$ are indecomposable and $\im \theta \neq 0$, it must be that either $V$ or $W$ is a summand of $\im \theta$ (since the decomposition into indecomposables is unique). If $\theta$ is neither injective or surjective, then by comparing dimensions we get a contradiction.
\end{proof}

\begin{lemma}
If $V \in \rep_\mathbb{F} Q$ is indecomposable, but not a brick, then $V$ has a subrepresentation which is a brick and has self-extensions (i.e. is not rigid). Dually, there is a quotient of $V$ which is a brick and not rigid.
\end{lemma} 

\begin{proof}
Since $V$ is not a brick, there exists a nonisomorphism, nonzero $\theta \in \End_Q(V)$. Choose such a $\theta$ so that $\im \theta$ has minimal dimension. Then $\theta(\im \theta)$ must be  proper subrepresentation of $\im \theta$ -- otherwise $\im \theta$ would be a summand of $V$. Since $\theta(\im \theta) = \im \theta^2$, by minimality of $\dim \im \theta$, this implies that $\theta(\im \theta) = 0$, i.e. $\im \theta \subseteq \ker \theta$, i.e. $\theta^2 = 0$. Consider a decomposition $\ker \theta = \oplus_i K_i$ into indecomposables. Pick a $j$ such that $\varphi: \im \theta \to \ker \theta \to K_j$ is nonzero. By minimality of $\im \theta$, $\varphi$ is injective. Now the bottom row of the following pushout diagram cannot split since $V$ is indecomposable:

\begin{tikzcd}
0 \ar[r] & \ker \theta \ar[r] \ar[d] & V \ar[r] \ar[d] & \im \theta \ar[r] \ar[d,equal] & 0 \\
0 \ar[r] & K_j \ar[r] & W \ar[r] & \im \theta \ar[r] & 0
\end{tikzcd}

That means that $\Ext(\im \theta, K_j) \neq 0$. Applying $\Ext(-,K_j)$ to the injection $\varphi$, we have a surjective map $\Ext(K_j, K_j) \twoheadrightarrow \Ext(\im \theta, K_j)$, and so $\Ext(K_j, K_J) \neq 0$, and $K_j$ is a proper subrepresentation of $V$.

If $K_j$ is not a brick, then we can iterate the construction. The yields a chain of proper subrepresentations of $V$, which must terminate, since $V$ was finite-dimensional.
\end{proof}

\begin{remark}
If $\mathbb{F}$ is algebraically closed, this also holds for infinite-dimensional representations, but we will not need this.
\end{remark} 

\begin{definition}
Let $Q$ be a quiver with no vertex loops. Then a vector $\underline d \in \mathbb Z_{\geq 0}^{Q_0}$ with $q(\underline d) = 1$ is called a \defterm{positive real root}.
\end{definition} 

\begin{lemma}
Let $V$ be a brick and $\underline{\dim} V$ is a positive real root. Then $V$ is rigid.
\end{lemma} 

\begin{terminology}
In this case, the vector $\underline\dim V$ is called a \defterm{real Schur root}. Classifying real Schur roots is a large open question, although an inductive description is known for all quivers.
\end{terminology} 

\begin{proof}
$1 = q(\underline \dim V) = \langle V, V\rangle = \dim \End_Q(V) - \dim \Ext_Q(V,V) = 1 - \dim \Ext_Q(V,V)$.
\end{proof} 

\begin{lemma}\label{lem:dynkin-brick-root}
 If $Q$ is Dynkin, then every indecomposable representation $V$ of $Q$ is a brick with $\underline \dim V$ a positive real root.
\end{lemma}

\begin{example}
Let $Q$ be the 2-Kronecker quiver,
\begin{tikzcd}
\mathbb{F} & \ar[l,shift left,"\mu"] \ar[l,shift right,"\lambda",swap] \mathbb{F}
\end{tikzcd}. Any representation $\delta_{[\lambda,\mu]}$ is a brick for any $[\lambda : \mu ] \in \mathbb P^1$, but $\Ext_Q(\delta_{[\lambda,\mu]}, \delta_{[\lambda,\mu]}) \cong \mathbb{F}$, so $(1,1)$ is \emph{not} a positive real root.
\end{example} 

\begin{proof}
If $V$ was not a brick, there would exist a brick $0 \neq U \subsetneq V$ with $\Ext(U,U) \neq 0$. But then $0 < q(\underline \dim U) = \dim \End_Q(U,U) - \dim \Ext_Q(U,U) = 1 - (>0) \leq 0$, a contradiction. The first, strict inequality, uses positive definiteness of $q$ since $Q$ is Dynkin.

It follows that $0 < q(\underline \dim V) = 1 - \dim \Ext_Q(V,V) = 1$ and so $\underline \dim V$ is a positive real root.
\end{proof}

\begin{corollary}\label{cor:dynkin-ind-dim}
For a Dynkin quiver, any two indecomposable representations with the same dimension vector are isomorphic.
\end{corollary}

\begin{proof}
By the results above, each will have a dense orbit in the moduli space of representations of the given dimension vector.
\end{proof}

\begin{lemma}
Suppose there is a non-split exact sequence $0 \to U \to V \to W \to 0$ $(\ast)$.

Then inside $M_{\underline \dim V} (Q)$, we have $\mathcal O_{W \oplus U} \subseteq \overline {\mathcal{O}}_V \setminus \mathcal O_V$.
\end{lemma}

\begin{proof}
Recall the map defining the Ext group: $\oplus_{\alpha \in Q_1} \Hom_{\mathbb F} (W_{s(\alpha)} U_{t(\alpha)}) \twoheadrightarrow \Ext_Q(W,U)$ is surjective. That is, by decomposing $V_i = W_i \oplus U_i$, we may write $V_\alpha = 1 \oplus f_\alpha$ where $f_\alpha: W_i \to U_j$.

For $\lambda \in \mathbb F^\times$, define $g_\lambda \in GL_{\underline \dim V} (\mathbb F) = \prod_{I \in Q_0} GL_{\dim V}(\mathbb F)$ by $g_{\lambda, i} = I_{W_i} \oplus \lambda I_{U_i}$, we have $(g_\lambda \cdot V)_\alpha = 1 +\lambda f_\alpha$. It follows that $\overline {\mathcal{O}}_V$ contains the representation at $\lambda=0$, which is just $W \oplus U$.

To see that $V \not \cong W \oplus U$, apply the functor $\Hom_Q(-,U)$ to the sequence $(\ast)$ to get the long exact sequence $0 \to \Hom_Q(W,U) \to \Hom_Q(V,U) \to \Hom_Q(U,U) \overset \delta \to \Ext_Q(W,U)$. Recall that $\delta$ is defined by pushing out our exact sequence along the map $U \to U$; in particular $\delta(I_U)$ is the class of the extension $(\ast)$ which is nonzero by assumption. Therefore $\dim \Hom(V,U) < \dim \Hom(W,U) + \dim \Hom(U,U)$, and therefore $V \not \cong W \oplus U$.
\end{proof}

\begin{corollary}
If $\mathcal O){W \oplus U} \subseteq M_{\underline d}(Q)$ is an orbit of maximal dimension, then $\Ext_Q(W,U) = 0$.
\end{corollary}

\begin{proof} Otherwise, there exists a non-split sequence as in $(\ast)$, and hence $\mathcal O_{W \oplus U} \subseteq \overline {\mathcal O}_V \setminus \mathcal O_V$ and so $\dim \mathcal O_{W \oplus V} < \dim \mathcal O_V$.
\end{proof}

\begin{corollary}\label{cor:dynkin-root-dim-ind}
For a Dynkin quiver $Q$ and a positive real root $\underline d \in \Phi_+$, there exists an indecomposable representation of $Q$ with dimension vector $\underline d$.
\end{corollary}

\begin{proof}
Consider $\mathcal O_V \subseteq M_{\underline d}(Q)$ of maximal dimension. If $V \cong W \oplus U$, then $\Ext_Q(W,U) = \Ext_Q(U,W) = 0$. But then $1 = q(\underline d) = \langle \underline d, \underline d \rangle = \langle W \oplus U, W \oplus U \rangle = q(\underline \dim W) + q(\underline \dim U) + 0 + 0 \geq 2$, a contradiction.
\end{proof} 

\begin{theorem}
If $Q$ is a Dynkin quiver, then the map $V \mapsto \underline \dim V$ gives a bijection between indecomposable representations of $Q$ and positive real roots for the quadratic form.
\end{theorem}

\begin{proof}
Lemma \ref{lem:dynkin-brick-root} gives the map; Corollary \ref{cor:dynkin-ind-dim} gives injectivity, and Corollary \ref{cor:dynkin-root-dim-ind} gives surjectivity.
\end{proof}

\begin{theorem}
A connected quiver $Q$ has finitely many isomorphism classes of indecomposable representations if and only if $Q$ is Dynkin.
\end{theorem}

\begin{proof}
We've seen that if $Q$ is Dynkin, then the indecomposable representations are in bijection with the set of postitive real roots, which is finite.

Conversely, recall from earlier that If $q(\underline d) \leq 0$, then there are infinitely many orbits in $M_{\underline d}(Q)$. But if there are finitely many indecomposables, then there are finitely orbits in $M_{\underline d}(Q)$. And only Dynkin quivers have positive definite $q$, even when restricted to nonnegative $\underline d$.
\end{proof}




  \section{Auslander-Reiten Theory}

Let $Q$ be a quiver with no vertex loops. For a vertex $i$, write $\sigma_i Q$ for the quiver obtained from $Q$ by reversing all arrows incident to vertex $i$. For each $\alpha \in Q_1$ with $s(\alpha) = i$ or $t(\alpha) = i$, there is a new arrow $\bar \alpha \in (\sigma_i Q)_1$ with $s(\bar \alpha) = t(\alpha)$ and $t(\bar \alpha) = s(\alpha)$. Q vertex $i \in Q_0$ is a \defterm{source} (resp. \defterm{sink}) if there does not exist an arrow $\alpha \in Q_1$ with $t(\alpha) = i$ (resp. $s(\alpha) = i$).

Definition. Let $i \in Q_0$ be a source of $Q$ and $V \in \Rep_{\mathbb F} Q$ Define a representation $\Sigma_i V \in \Rep_{\mathbb F} \sigma_i Q$ as follows.

$(\Sigma_i V)_j = V_j$ for $j \neq i$

$(\Sigma V)_\alpha = V_\alpha$ when $s(\alpha) =\neq i$

$(\Sigma_i V)_i = \coker(\oplus_{\alpha \in Q_1,s(\alpha) = i} V_\alpha: V_i \to \oplus_{\alpha \in Q_1, s(\alpha) = i} V_{t(\alpha)})$

For $\alpha \in Q_1$ with $s(\alpha) = i$, the map $(\Sigma_i V)_{\bar \alpha}$ is given by $\pi \iota_{t(\alpha)}$ where $\pi: \oplus_{\alpha \in Q_1, s(\alpha) = i} V_{t(\alpha)} \to (\Sigma_i V)_i$ is the natrual projection and $\iota_{t(\alpha)}: V_{t(\alpha)} \to \oplus_{\alpha \in Q_1,s(\alpha) = i} V_{t(\alpha)}$ is the inclusion of the $t(\alpha)$ component.

\begin{example}
Let $Q = 1 \leftarrow 2$ Then $\sigma_2 Q = 1 \to 2$. Recall that $S_1 = \mathbb F \leftarrow 0$. Then $\Sigma_1(S_1) = \mathbb F \overset 1 \to \mathbb F$. Recall that $S_2 = 0 \leftarrow \mathbb F$. Then $\Sigma_2(S_2) = 0 \to 0$. Recall that $P_2 = \mathbb F \leftarrow \mathbb F$. Then $\Sigma_2(P_2) = \mathbb F \to 0$.
\end{example} 

%%%%%%%%%%%%%%%%%%%%%%%
\chapter{Hall Algebras}

  \section{Basic Definitions}
  
      	Let Q be a quiver and $\mathbb{F}$ a field.
 	\underline{Goal}: Encode the extension structure of the category $\rep_\mathbb{F}Q$ into an algebraic object and then use our understanding of $\rep_\mathbb{F}Q$ to study this object.
    
    	\subsection{Geometric Picture}
        	Recall the moduli space $M_{\underline{d}}(Q)$ of representations of dimension vector $\underline{d}\in \mathbb{Z}_{\geq 0}^{Q_0}$. For $\underline{d},\underline{d}'\in \mathbb{Z}_{\geq 0}^{Q_0}$, write $E_{\underline{d},\underline{d}'}(Q)$ for the set of all short exact sequences
            		\[ \begin{tikzcd}
				0 \arrow{r} & U \arrow{r} &V \arrow{r} &W \arrow{r} &0
			\end{tikzcd}\]
		with $W\in M_{\underline{d}}(Q)$ and $U\in M_{\underline{d}'}(Q)$. There exist projections
			\[ \begin{tikzcd} &\arrow{dl}{\pi_1} E_{\underline{d},\underline{d}'}(Q)\arrow{d}{\pi_2}\arrow{dr}{\pi_3}& \\
		    		M_{\underline{d}}(Q) & M_{\underline{d}+\underline{d}'}(Q) & M_{\underline{d}'}(Q)
			\end{tikzcd}\]
        	Given functions $f:M_{\underline{d}}(Q)\to \mathbb{C}$ and $g:M_{\underline{d}'}(Q)\to \mathbb{C}$, we can pull each of these back to $E_{\underline{d},\underline{d}'}(Q)$ and multiply them there $\pi_1^*(f)\pi_3^*(g)$ then push-forward along the map $\pi_2$ by integrating along the fibers:
        	\[(\pi_2)_*\left(\pi_1^*(f)\pi_3^*(g)\right)(v)=\int_{(w,u)\in \pi_2^{-1}(v)}f(w)g(w) \ d\chi\]
        	We won't make precise what measure to use in general, but we will use this as motivation for defining an algebra attached to $Q$ whose structure constants encode extensions.
    
    \begin{definition}
        For $W\in M_{\underline{d}}(Q)$, write $1_W$ for the characteristic function
        \[1_W(W') = \begin{cases} 
        1 & \text{if } \  W=W' \\
        0 & \text{else}
        \end{cases}\]
    \end{definition}
    
    \textbf{Observation}: When computing $(\pi_2)_*(\pi_1^*(1_W)\pi_3^*(1_U))(v)$ on a point $V\in M_{\underline{d}+\underline{d}'}(Q)$, we only have to integrate over the space 
    \begin{align*}
        \mathcal{P}_{W,U}^V :&= \pi_2^{-1}(V)\cap\pi_1^{-1}(W)\cap \pi_3^{-1}(U)\\
        &= \left\{ (\theta,\varphi): 
        0\to U\xrightarrow{\theta} V\xrightarrow{\varphi} W\to 0
        \ \text{is exact}\right\}
    \end{align*}
    
    From now on, take $\mathbb{F}=\mathbb{F}_q$ a finite field with $q$ elements. Define the measure $\chi$ on $E_{\underline{d},\underline{d}'}(Q)$ to be the weighted counting measure with 
    \[\chi\left(0\to U\to  V\to W\to 0\right):=\frac{q^{\left<W,U\right>}}{a_W \cdot a_U}=\frac{q^{\left<\underline{d},\underline{d}'\right>}}{a_W\cdot a_U} \]
    where $A_W=|\text{Aut}(W)|$.
    
    \begin{definition}
        Write $\mathcal{H}_Q$ for the space of finitely supported functions on \[\prod_{\underline{d}\in\mathbb{Z}_{\geq 0}^{Q_0}} M_{\underline{d}}(Q).\] This is spanned by the characteristic functions $1_W$ for $W\in M_{\underline{d}}(Q)$ and we can consider $\mathcal{H}_Q$ as an algebra using the convolution product
        \begin{align*}
            1_W*1_U &= (\pi_2)_*(\pi_1^*1_W\cdot \pi_3^*1_U)\\
            &= \sum_{V\in M_{\underline{d}+\underline{d}'}(Q)}q^{\left<\underline{d},\underline{d}'\right>}\frac{|\mathcal{P}_{W,U}^V|}{a_W\cdot a_U}1_V.
        \end{align*}
    \end{definition}
    
     Fix $V$ such that $t^2=q$. Then $\mathcal{H}_Q$ is the $\mathbb{C}$-vector space spanned by the isomorphism classes of representations in $\rep_\mathbb{F}(Q)$ with multiplication given by
    \[[W]\cdot[U]=\sum_{[V]}t^{\left< W,U \right> }\frac{|\mathcal{P}_{W,U}^V|}{a_W\cdot a_U}[V].\] 
    
    \begin{lemma}
    	\[\frac{|\mathcal{P}_{W,U}^V|}{a_W\cdot a_U}=|\mathcal{F}_{W,U}^V|\] where \[\mathcal{F}_{W,U}^V=\{E\subseteq V : E\cong U, V/E \cong W\}\]
    \end{lemma}
    \begin{proof}
    	There exists a map $\pi:\mathcal{P}_{W,U}^V\to\mathcal{F}_{W,U}^V$ which sends $(\theta,\varphi)$ to $\im(\theta)$. The fiber of $\pi$ over $E\in \mathcal{F}_{W,U}^V$ is $\{(\theta,\varphi):\im\theta = E\}\cong\text{Aut}(W)\times\text{Aut}(U)$.
    \end{proof}
    
    Let $P_{W,U}^V=|\mathcal{P}_{W,U}^V|$ and $F_{W,U}^V=|\mathcal{F}_{W,U}^V|$.
    
    \begin{corollary}
        The multiplication on $\mathcal{H}_Q$ is associative with unit $[0]$.
    \end{corollary}
    
    \begin{proof}
        Unit is obvious. For $U,W,X\in\rep_\mathbb{F}(Q)$, we have 
        \[[X]\cdot\Big([W]\cdot[U]\Big)=\sum_{[V]}q^{\left<W,U\right>}F_{W,U}^V[X]\cdot[V]=\sum_{[Y]}\sum_{[V]}q^{\left<W,U\right>}q^{\left<X,V\right>}F_{W,U}^VF_{X,V}^Y[Y]\]
        and
        \[ \Big([X]\cdot[W]\Big)\cdot[U] = \sum_{[T]}q^{\left<X,W\right>}F_{X,W}^T[T]\cdot[U] = \sum_{[Y]}\sum_{[T]}q^{\left<X,W\right>}q^{\left<T,U\right>}F_{X,W}^T F_{T,U}^Y[Y].\]
        But we get \[ \left<W,U\right>+\left<X,V\right> = \left<W,U\right>+\left<X,W\right>+\left<X,U\right> = \left<X,W\right>+\left<T,U\right>.\]
        So we need to check that \[\sum_{[V]}F_{W,U}^V F_{X,V}^Y=\sum_{[T]}F_{X,W}^T F_{T,U}^Y.\]
        Using the lemma, we see that $\sum_{[V]}F_{W,U}^V F_{X,V}^Y$ counts the cardinality of the set 
        \[\{(A,B): A\subseteq B, B\subseteq Y, A\cong U, B/A\cong W, Y/B\cong X\}\]
        and that $\sum_{[T]}F_{X,W}^T F_{T,U}^Y$ counts the cardinality of the set 
         \[\{(C,A): A\subseteq Y, C\subseteq Y/A, A\cong U, C\cong W, (Y/A)/C\cong X\}.\]
         But observe that choosing $C\subseteq Y/A$ is the same as choosing intermediate representation $A\subseteq B\subseteq Y$, where the bijection is given by $C=B/A$. Finally, $(Y/A)/C=(Y/A)/(B/A)=Y/B$, so the conditions on the sets are the same and so this bijection induces a bijection of the sets above.
    \end{proof}
    
    \subsection{Combinatorics of Finite Linear Algebra}
    Let $Q$ be a quiver with a single vertex. (More generally, fix attention on a single vertex of an arbitrary quiver.) Let $S\in \rep_{\mathbb{F}}Q$ be the unique simple representation. Our goal is to compute the multiplication in $\mathcal{H}_Q$ completely, i.e. understand $[S]^n$.
    
    \bigskip\noindent
    \textbf{Observations}: 
        \begin{enumerate}
            \item \begin{align*} |\text{Aut}(S^n)|&=|\text{GL}_n(\mathbb{F}_q)|\\
            &=(q^n-1)(q^n-q)(q^n-q^2)\cdots(q^n-q^{n-1})\\
            &= q^{\binom{n}{2}}(q-1)^n[n]^!_q
            \end{align*} where $[k]_q=q^{k-1}+q^{k-2}+\ldots+q+1$ and $[n]_q^! = [n]_q\cdot [n-1]_q\cdots [2]_q\cdot [1]_q$.
            \item \begin{align*} F_{S^{n-k}S^k}^{S^n}&=|\{E\subseteq S^n : E\cong S^k, S^n/E\cong S^{n-k}\}|\\
            &=|\text{Gr}_k(\mathbb{F}_q^n)|\\
            &= \frac{(q^n-1)(q^n-q)\cdots(q^n-q^{k-1})}{|\text{Aut}(S^k)|}\\
            &= \frac{q^{\binom{k}{2}}(q-1)^k[n]_q[n-1]_q \cdots [n-k+1]_q}{q^{\binom{k}{2}} (q-1)^k[k]_q^!}\\
            &= \frac{[n]_q^!}{[k]_q^![n-k]_q^!}=:\begin{bmatrix}n\\ k\end{bmatrix}_q
            \end{align*}
            We call $\begin{bmatrix}n\\ k\end{bmatrix}_q$ the \textbf{quantum binomial coefficient}.
        \end{enumerate}
	
Let \[(n)_t=t^{-n+1}+t^{-n+3}+\ldots+t^{n-3}+t^{n-1}.\] Then 
    \begin{align*}
        \left[S^{n-k}\right]\left[S^k\right]&=t^{k(n-k)}\begin{bmatrix}n\\ k\end{bmatrix}_q\left[S^n\right]\\
        [S]^n &= t^{\binom{n}{2}}[n]_q^!\left[S^n\right]\\
        [S]^{(n)} &= \frac{[S]^n}{(n)_t^!}=t^{n(n-1)}\left[S^n\right].
    \end{align*}
    
     \begin{lemma}
        Let $X$ and $Y$ be noncommuting variables with $YX=t^2XY$. Then \begin{align*}
            (X+Y)^n&=\sum_{k=0}^n \begin{bmatrix}n\\ k\end{bmatrix}_q X^kY^{n-k}\\
            &= \sum_{k=0}^n t^{-k(n-k)}\begin{bmatrix}n\\ k\end{bmatrix}_q t^{k(n-k)}X^kY^{n-k}\\
            &= \sum_{k=0}^n \binom{n}{k}_t \  t^{k(n-k)}X^kY^{n-k}
            \end{align*}
    \end{lemma}
    
    \begin{proof}
        Observe that \[\begin{bmatrix}n\\ k\end{bmatrix}_q=q^{n-k}\begin{bmatrix}n-1\\ k-1\end{bmatrix}_q+\begin{bmatrix}n-1\\ k\end{bmatrix}_q = \begin{bmatrix}n-1\\ k-1\end{bmatrix}_q+q^k\begin{bmatrix}n-1\\ k\end{bmatrix}_q\] and then work by induction.
    \end{proof}
    
    \begin{corollary}
        \[\sum_{k=0}^n (-1)^k t^{k(k-1)}\begin{bmatrix}n\\ k\end{bmatrix}_q=0.\]
    \end{corollary}
    
    \begin{proof}
        \[(X+Y)^n=\left[(1+XY^{-1})Y\right]^n=(1+XY^{-1})(1+t^2XY^{-1})\cdots(1+t^{2(n-1)}XY^{-1})Y^n\]
        and \[(XY^{-1})^k=t^{-k(k-1)}X^kY^{-k}.\] So for $Z=XY^{-1}$ we have 
        \[\prod_{k=0}^{n-1} (t+t^{2k}Z)=\sum_{k=0}^nt^{k(k-1}\begin{bmatrix}n\\ k\end{bmatrix}_qZ^k.\]
        Take $Z=-1$ to get \[0=\sum_{k=0}^n (-1)^k t^{k(k-1)}\begin{bmatrix}n\\ k\end{bmatrix}_q.\]
    \end{proof}
    
    \begin{example}[Rank 2]
        Consider the $n$-Kronecker quiver \[ Q= \begin{tikzcd}
			1  &   \arrow[l, "\alpha_1"',yshift=1.5ex]  \arrow[l, phantom, "{\vdots}"',yshift=.5ex]     \arrow[l,"\alpha_n",yshift=-1.5ex] 2
			\end{tikzcd} \]
    \end{example}
    
    \begin{lemma}\label{(n+1,1)}
		    Any representation of $Q$ with dimension vector $(n+1,1)$ or $(1,n+1)$ is decomposable.
	\end{lemma}
	
	\begin{proof}
	    Let $V=(V_i,V_\alpha)$ be a representation of $Q$ with $\underline{\dim}V=(n+1,1)$. The total map \[\sum_{i=1}^nV_{\alpha_i}:\bigoplus_{i=1}^nV_2\to V_1\] has at most an $n$-dimensional image and so $V$ admits a summand isomorphic to $S_1$. The other case is dual.
	\end{proof}
    
    \begin{lemma}
        Let $V$ be an indecomposable representation of $Q$ with $\underline{\dim}V=(a,1)$. Then for any $s\geq 0$ we have $F_{S_1^s V}^{S_1^s\oplus V}=1.$ Similarly, if $\underline{\dim}V=(1,a)$, then  $F_{V S_2^s}^{V\oplus S_2^s}=1$ for any $s\geq 0$.
    \end{lemma}
    
    \begin{proof}
        Any subrepresentation $E\subseteq S_1^s\oplus V$ with $E\cong V$ gives an exact sequence 
        \[\begin{tikzcd}
            0\arrow{r} & \mathbb{F} \arrow[d,xshift=1.6ex]  \arrow[d, phantom, "{\hdots}"']     \arrow[d,xshift=-1.6ex]\arrow{r}{\id} & \mathbb{F}\oplus0 \arrow[d,xshift=-1.5ex] 
            \arrow[d, phantom, "{\oplus}"'] \arrow[d, phantom, "{..}"',xshift=-2ex]     \arrow[d,xshift=-2.6ex] \arrow[d,xshift=1.5ex]  \arrow[d, phantom, "{..}"',xshift=2ex]     \arrow[d,xshift=2.6ex] \arrow{r} & 0 \arrow{d}\arrow{r} & 0 \\
            0\arrow{r} & \mathbb{F}^a \arrow{r}{f} & \mathbb{F}^a\oplus \mathbb{F}^s \arrow{r} & \mathbb{F}^s \arrow{r} & 0
        \end{tikzcd}\]
        where the left hand vertical morphisms are just the restrictions of the middle vertical morphisms. In particular, the image of $f$ inside $\mathbb{F}^a\oplus \mathbb{F}^s$ is uniquely determined.
    \end{proof}
    
    \begin{corollary}
        For $V$ indecomposable with 
        \begin{enumerate}
            \item ${\underline \dim V} = (a,1)$, we have $\left[S_1^s \right][V]=t^{sa}\left[S_1^s\oplus V\right]$
            \item ${\underline \dim V} = (1,a)$, we have $[V]\left[S_2^s \right]=t^{sa}\left[V\oplus S_2^s\right]$.
        \end{enumerate}
    \end{corollary}
    
    \begin{proof}
        $\left<S_1^s,V\right>=sa$ or $\left<V,S_2^s\right>=sa$.
    \end{proof}
    
    \begin{lemma}
        For any $r,s\geq 0$, \[\left[S_2\right]\left[S_1^s\right]=\sum_{\substack{[V]\\ \underline{\dim}V=(s,1)}}t^{-ns}[V]\] and \[\left[S_2^s\right]\left[S_1\right]=\sum_{\substack{[V]\\ \underline{\dim}V=(1,r)}}t^{-nr}[V].\]
    \end{lemma}
    
    \begin{proof}
        $\left<S_2,S_1^s\right>=-ns$ and for any $V$ with $\underline{\dim}V=(s,1)$, there exists a unique subrepresentation isomorphic to $S_2$.
    \end{proof}
    
    For $a\geq 0$, write \[X_a = \sum_{\substack{V\ \text{indecomposable} \\ \underline{\dim}V=(a,1)}}[V]\] and \[Y_a = \sum_{\substack{V\ \text{indecomposable} \\ \underline{\dim}V=(1,a)}}[V].\] By \eqref{(n+1,1)}, these are zero for $a>n$.
    
    \begin{corollary}
        For any $r,s\geq 0$, we have 
        \[\left[S_1^r\right]\left[S_2\right]\left[S_1^s\right]  = \sum_{a=0}^s t^{-ns+(r-a)(s-a)+rs}\begin{bmatrix}r+s-a\\ r\end{bmatrix}_q \left[S_1^{r+s-a}\right]X_a \]
            and \[\left[S_2^r\right]\left[S_1\right]\left[S_2^s\right]  = \sum_{a=0}^r t^{-nr+(r-a)(s-a)+rs}\begin{bmatrix}r+s-a\\ s\end{bmatrix}_q Y_a\left[S_2^{r+s-a}\right].\]
    \end{corollary}
   
    Let $Q$ be a quiver and fix $i,j\in Q_0$ so that these vertices are joined by $n$ arrows from $j$ to $i$. Then
    \begin{align*}
      \left[S_i\right]^k & =  t^{\binom{k}{2}}[k]_q^!\left[S_i^k\right]\\
       \left[S_i^k\right]\left[S_i^l\right] & =  t^{lk}\begin{bmatrix}l+k\\ k\end{bmatrix}_q\left[S_i^{k+l}\right]
    \end{align*}
    If $\dim V_i=a$, $\dim V_j=1$, and $\dim V_k=0$ for $k\neq i,j$, then 
    \[\left[S_i^r\right]\left[V\right]  =  t^{ra}\left[S_i^r\oplus V\right].\]
    If $\dim V_i=s$, $\dim V_j=1$, and $\dim V_k=0$ for $k\neq i,j$, then 
    \[\left[S_j\right]\left[S_i^r\right]  =  \sum_{[V]}t^{-nr}\left[V\right].\]
   
    \begin{theorem}[Ringel]
        Inside $\mathcal{H}_Q$, we have \[\sum_{r=0}^{n+1}(-1)^r[S_i]^{(r)}[S_j][S_i]^{(n+1-r)}=0\] and \[\sum_{r=0}^{n+1}(-1)^r[S_j]^{(r)}[S_i][S_j]^{(n+1-r)}=0.\]
    \end{theorem}
    
    

         \begin{proof}
        \begin{align*}
            \sum_{r=0}^{n+1}(-1)^r & t^{r(r-1)+(n+1-r)(n-r)}[S_i^r][S_j][S_i^{n+1-r}]\\
            &=\sum_{r=0}^{n+1}(-1)^rt^{r(r-1)+(n+1-r)(n-r)}\sum_{\substack{[V]\\ \dim V_i = n+1-r \\ \dim V_j =1\\ \dim V_k=0 \ k\neq i,j}}t^{-n(n+1-r)}[S_i^r][V] \\
            &=\sum_{r=0}^{n+1}(-1)^rt^{r(r-1)+(n+1-r)(n-r)}\sum_{a=0}^{n+1-r}t^{-n(n+1-r)}t^{-a(n+1-r-a)}[S_i^r][S_i^{n+1-r-a}]X_a \\
            &=\sum_{r=0}^{n+1}(-1)^rt^{r(r-1)+(n+1-r)(n-r)}\sum_{a=0}^{n+1-r}t^{-n(n+1-r)}t^{-a(n+1-r-a)}t^{r(n+1-r-a)}\begin{bmatrix}n+1-a\\ r\end{bmatrix}_q [S_i^{n+1-a}]X_a\\
            &=\sum_{a=0}^{n+1}\left(\sum_{r=0}^{n+1-a}(-1)^rt^{r(r-1)}\begin{bmatrix}n+1-a\\ r\end{bmatrix}_q\right) t^{-a(n+1-a)}[S_i^{n+1-a}]X_a\\
            &=0
        \end{align*}   
        where \[X_a=\sum_{\substack{W indec.\\ \dim W_i=q\\\dim W_j=1}}[W].\] The proof of the other is similar.
    \end{proof}

    These are called the \textbf{quantum Serre relations}.
    
    \begin{definition}
        Let $\mathcal{C}_Q\subseteq \mathcal{H}_Q$ denote the \textbf{composition subalgebra} generated over $\mathbb{Z}[t^{\pm 1}]$ by the divided powers $[S_i]^{(r)}$ for $i\in Q_0$ and $r\geq 0$.
    \end{definition}
    
    \underline{Goal}: Show that the quantum Serre relations determine the composition subalgebra.
    
    \subsection{Coalgebras}
        Recall: An associative and unital \textbf{algebra} (over $\mathbb{C}$) is a vector space together with linear maps $\eta:\mathbb{C}\to A$ (unit) and $\mu:A \otimes A\to A$ (multiplication) such that the following diagrams commute:
        \[\begin{tikzcd}
            A\otimes A\otimes A \arrow{r}{\mu\otimes 1}\arrow{d}{1\otimes\mu} & A\otimes A \arrow{d}{\mu} & & A\cong \mathbb{C}\otimes A \arrow{d}{1\otimes\eta}\arrow{r}{\eta\otimes 1}\arrow{dr}{\id}& A\otimes A \arrow{d}{\mu}
            \\ A\otimes A \arrow{r}{\mu}& A & & A\otimes A\arrow{r}{\mu} & A
        \end{tikzcd}\]
        A homomorphism of algebras is a map $\phi:A\to A'$ such that the following diagrams commute:
        \[\begin{tikzcd}
            &A \arrow{dd}{\phi}
            & & A\otimes A \arrow{d}{\phi\otimes\phi}\arrow{r}{\mu}& A \arrow{d}{\phi}
            \\ \mathbb{C}\arrow{ur}{\eta}\arrow{dr}{\eta'}&
            & & A'\otimes A'\arrow{r}{\mu'} & A'\\
            &A'
        \end{tikzcd}\]
        The dual of a finite dimensional algebra is a \textbf{coalgebra}: a vector space $C$ together with linear maps $\epsilon:C\to\mathbb{C}$ (counit) and $\Delta:C\to C\otimes C$ (comultiplication) such that the following diagrams commute:
        \[\begin{tikzcd}
            C \arrow{r}{\Delta}\arrow{d}{\Delta} & C\otimes C \arrow{d}{\Delta\otimes 1} 
            & & C \arrow{dr}{\id} \arrow{r}{\Delta}\arrow{d}{\Delta} & C\otimes C \arrow{d}{\epsilon\otimes 1}
            \\ C\otimes C \arrow{r}{1\otimes\Delta}& C\otimes C\otimes C 
            & & C\otimes C\arrow{r}{1\otimes\epsilon} & C\cong \mathbb{C}\otimes C\cong C\otimes\mathbb{C}
        \end{tikzcd}\]
        A homomorphism of coalgebras is a map $\psi:C\to C'$ such that the following diagrams commute:
        \[\begin{tikzcd}
            C\arrow{dr}{\epsilon}\arrow{dd}{\psi}& 
            & & C \arrow{d}{\psi}\arrow{r}{\Delta}& C\otimes C \arrow{d}{\psi\otimes\psi}
            \\ &\mathbb{C}
            & & C'\arrow{r}{\Delta'} & C'\otimes C'\\
            C'\arrow{ur}{\epsilon'}&
        \end{tikzcd}\]
        
        \begin{exercise}
            Let $B$ be a vector space together with linear maps $\eta:\mathbb{C}\to B$, $\mu:B\otimes B\to B$, $\epsilon:B\to\mathbb{C}$, and $\Delta:B\to B\otimes B$ such that $(B,\eta,\mu)$ is an algebra and $(B,\epsilon,\Delta)$ is a coalgebra. Consider $B\otimes B$ as an algebra with multiplication $\tilde{\mu}:(B\otimes B)\otimes (B\otimes B)\to B\otimes B $ given by $(\mu\otimes \mu)\circ(1 \otimes \tau\otimes 1)$ where $\tau:B\otimes B\to B\otimes B$ interchanges factors, and as a coalgebra with comultiplication $\Delta\otimes\Delta:B\otimes B \to (B\otimes B)\otimes(B\otimes B)$. Show that $\mu$ is a coalgebra morphism if and only if $\Delta$ is an algebra morphism. 
        \end{exercise}
        
        \begin{definition}
            In the situation above, we call $(B,\eta,\mu,\epsilon,\Delta)$ a \textbf{bialgebra} if the following diagram commutes:
            \[\begin{tikzcd}
            B\otimes B && \arrow{ll}{\Delta}B\arrow{dr}{\epsilon} && B\otimes B \arrow{ll}{\mu}\arrow{dl}{\epsilon\otimes\epsilon} \\
            & \mathbb{C}\arrow{ul}{\eta\otimes\eta}\arrow{ur}{\eta}\arrow{rr}{\id} && \mathbb{C} &
            \end{tikzcd}\]
        \end{definition}

\section{Bialgebra Structure}

\begin{definition} Define a map $\Delta : \mathcal{H}_Q \rightarrow \mathcal{H}_Q \times \mathcal{H}_Q$ by $$\Delta([V])=\sum_{[U],[W]} t^{<W,U>} F^V_{W,U} \frac{a_u a_w}{a_v} [W][U]=\sum_{[U],[W]} t^{<W,U>} P^V_{W,U} \frac{1}{a_v} [W][U]$$ and a map $\epsilon : \mathcal{H}_Q \rightarrow \mathbb{C}$ by $$\epsilon ([V]) =   \begin{Bmatrix} 1 & if & V=0 \\ 0 & else & \end{Bmatrix} $$
\end{definition}

\begin{prop}
$(\mathcal{H}_Q, \epsilon, \Delta)$ is a (counital coassociative) coalgebra. 
\end{prop}

\begin{proof}
The identities $(\epsilon \otimes 1) \Delta ([V]) = [V] = (1 \otimes \epsilon) \Delta ([V])$ are clear because $<V,0>=0=<0,V>$. Thus we compute 
\begin{align*} 
 (\Delta \otimes 1)\Delta ([Y])&=  \sum_{[U],[T]} t^{<T,U>} F^Y_{T,U} \frac{a_T a_U}{a_Y}\Delta( [T] )\otimes [U] \\ 
 &=  \sum_{[U],[T],[W],[X]} t^{<T,U>+<X,W>} F^Y_{X,V}  F^V_{W,U} \frac{a_U a_W a_X}{a_Y}[X] \otimes [W] \otimes [U] 
\end{align*}
and 
\begin{align*} 
 (1 \otimes \Delta)\Delta ([Y])&=  \sum_{[V],[X]} t^{<X,V>} F^Y_{X,V} \frac{a_X a_V}{a_Y} [X] \otimes \Delta( [U]) \\ 
 &=  \sum_{[U],[T],[W],[X]} t^{<X,Y>+<W,U>} F^Y_{X,V}  F^V_{W,U} \frac{a_U a_W a_X}{a_Y}[X] \otimes [W] \otimes [U] 
\end{align*}
Note that $<T,U>=<X+W,U>$ and $<X,V>=<X,W+U>$.
So we need that $$\sum_{[T]} F^Y_{T,U} F^T_{X,W} = \sum_{[V]} F^Y_{X,V} F^V_{W,U}$$ but this is exactly the identity appearing in the associativity of the multiplication on $\mathcal{H}_Q$. \\ So $(\Delta \otimes 1)\Delta = (1 \otimes \Delta)\Delta.$
\end{proof}

We now can ask if the multiplication and comultiplication on $\mathcal{H}_Q$ are compatible. 

\begin{example}
Q: $1 \leftarrow 2$. Take $W=S_2$, $U=S_1$, $P_2= \mathbb{F} \xleftarrow{id} \mathbb{F}$.\\
Then 

\begin{align*} 
[S_2][S_1] &= t^{-1}([P_2]+[S_1 \oplus S_2]) \\
\Delta ([S_2]) &= 1 \otimes [S_2]+ [S_2] \otimes 1 \text{  (primitive)} \\
\Delta ([S_1]) &= 1 \otimes [S_1]+ [S_1] \otimes 1 \\
\Delta ([P_2]) &= 1 \otimes [P_2]+ [P_2] \otimes 1 + t^{-1} (q-1) [S_2] \otimes [S_1]  =  1 \otimes [P_2]+ [P_2] \otimes 1 + (t-t^{-1} ) [S_2] \otimes [S_1]\\
\Delta ([S_1\oplus S_2]) &=1 \otimes [S_1\oplus S_2] + [S_1\oplus S_2] \otimes 1 + [S_1] \otimes [S_2]  + t^{-1} [S_2] \otimes [S_1] \\
\Delta ([S_2][S_1]) &=t^{-1}(1 \otimes [P_2] + [P_2] \otimes 1 + [S_1\oplus S_2] \otimes 1  +  1\otimes  [S_1\oplus S_2] + [S_1] \otimes [S_2]) +[S_2] \otimes [S_1] \\
\Delta ([S_2])\Delta([S_1]) &=t^{-1}( [P_2] +[S_1\oplus S_2]) \otimes 1 + 1 \otimes t^{-1} ( [P_2] +[S_1\oplus S_2]) +  [S_2] \otimes [S_1] +[S_1] \otimes [S_2] 
\end{align*}
So we can see that $\Delta ([W][U]) \neq \Delta([W])\Delta([U])$
\end{example}

We have a solution to this problem. We twist the algebra structure on $\mathcal{H}_Q \otimes \mathcal{H}_Q$ using the pairing $(W,U)=<W,U>+<U,W>$. 
$$([W]\otimes [U])([W']\otimes[U'])=t^{([U],[W'])}[W][W'] \otimes [U][U']$$

\begin{theorem}[Green]
Considering $\mathcal{H}_Q \otimes \mathcal{H}_Q$ with this twisted multiplication, the map $\Delta: \mathcal{H}_Q  \rightarrow \mathcal{H}_Q \otimes \mathcal{H}_Q$ is an algebra homomorphism. 
\end{theorem}


To prove the theorem we need to show for $W,U \in rep_{\mathbb{F}}Q$ we need to check that $\Delta ([W][U])=\Delta ([W])\Delta ([U])$ 

$$\Delta ([W][U]) =  \sum_{[V]} t^{<W,U>} F^V_{W,U} \Delta( [V] ) =  \sum_{[V],[X],[Y]} t^{<W,U>+<Y,X>} F^V_{W,U}  F^V_{Y,X} \frac{a_X a_Y}{a_V} [Y] \otimes [X]$$

\begin{align*} 
\Delta ([W]) \Delta([U]) &=  \left(\sum_{[C],[D]} t^{<D,C>} F^W_{D,C}  \frac{a_C a_D}{a_W} [D] \otimes [C]\right) \left(\sum_{[A],[B]} t^{<B,A>} F^U_{B,A}  \frac{a_A a_B}{a_U} [B] \otimes [A] \right) \\
&= \sum_{[A],[B],[C],[D]} t^{<D,C>+<B,A>+([C],[B])} F^U_{B,A}  F^W_{D,C} \frac{a_A a_B a_C a_D}{a_U a_W} [D][B] \otimes [C][A]\\
&=  \sum_{[A],[B],[C],[D],[X],[Y]} t^{<D,C>+<B,A>+([C],[B])+<D,B>+<C,A>} F^U_{B,A}  F^W_{D,C} F^Y_{D,B} F^X_{C,A} \frac{a_A a_B a_C a_D}{a_U a_W} [Y] \otimes [X]
\end{align*}
Note that $<W,U>+<Y,X>=<C+D,A+B>+<B+D,A+C> = 2<D,A> +(\text{exponent on second sum}).$
So we need to show
$$a_Ua_Wa_Xa_Y \sum_{[V]} F^V_{W,U} F^V_{Y,X}a_V^{-1} = \sum_{[A],[B],[C],[D]}q^{-<D,A>} F^U_{B,A} F^W_{D,C} F^Y_{D,B} F^X_{C,A} a_Aa_Ba_Ca_D$$  
This equation is called Green's Formula. \\
The left hand side counts "crosses" weighted by $a_V^{-1}=|Aut(V)|^{-1}.$ 
 \[\begin{tikzcd}
             & & 0\arrow{d}& &  \\
	 & & X\arrow{d}& &  \\
	0 \arrow{r} & U\arrow{r} &V \arrow{d} \arrow{r}& W \arrow{r}& 0 \\
           & & Y\arrow{d}& &  \\
	& & 0& &  
            \end{tikzcd}\]
The right hand side counts "boxes" weighted by $a_Aa_Ba_Ca_D q^{-<D,A>}$
 \[\begin{tikzcd}
             &  0\arrow{d}& &0\arrow{d} &  \\
	0 \arrow{r} & A \arrow{r}{\iota_X} \arrow{d}{\iota_U}& X\arrow{r}{\pi_X}& C\arrow{r}\arrow{d}{\iota_W}& 0 \\
	& U\arrow{d}{\pi_U} & & W \arrow{d}{\pi_W}&  \\
           0 \arrow{r} & B \arrow{r}{\iota_Y} \arrow{d}& Y\arrow{r}{\pi_Y}& D\arrow{r}\arrow{d}& 0 \\
	& 0 & & 0&  
            \end{tikzcd}\]
The goal is to give a weighted bijection between the collections of such configurations. 
\begin{lemma}
Given a box as above define M and N by the following diagrams:\\
\begin{center}
 $\begin{tikzcd}
            A \arrow{r}{\iota_X} \arrow{d}{\iota_U}  \arrow[dr, phantom, "\ulcorner", very near start] & X \arrow{d}{\alpha_X} \\
	U \arrow{r}{\alpha_U} &  M
            \end{tikzcd}$
\hskip 10 pt 
 $\begin{tikzcd}
            N \arrow{r}{\beta_W} \arrow{d}{\beta_Y}  \arrow[dr, phantom, "\lrcorner", very near end] & W \arrow{d}{\pi_W} \\
	Y \arrow{r}{\pi_Y} &  D
            \end{tikzcd}$
\end{center}
There exists a unique map $\theta : M \to N$ such that:
\begin{align*} 
\beta_Y \circ \theta \circ \alpha_X &=0 & \beta_Y \circ \theta \circ \alpha_U &= \iota_Y \circ \pi_Y     \\
 \beta_W \circ \theta \circ \alpha_X &= \iota_W \circ \pi_X  &   \beta_W \circ \theta \circ \alpha_U &=0
\end{align*}

This map gives an exact sequence:
$$0 \rightarrow A \xrightarrow[\bullet=X \text{ or } U]{\alpha_\bullet \circ \iota_\bullet} M \xrightarrow{\theta} N \xrightarrow[\bullet=Y \text{ or } W]{\pi_\bullet \circ \beta_\bullet} D \rightarrow 0$$

\end{lemma}

\begin{proof}(of lemma) Consider the zero map $0:X \rightarrow Y$ together with the map $\iota_Y \circ \pi_U : U \rightarrow Y$. These satisfy $0\circ \iota_X = 0 = \iota_Y \circ \pi_U \circ \iota_U$ and so there is a map $\theta_Y:M\rightarrow Y $ such that $\theta_Y \circ \alpha_X = 0$ and $\theta_Y \circ \alpha_U = \iota_Y \circ \pi_U$. Similarly, using zero map $0:U \rightarrow W$ and the map $\iota_W \circ \pi_X :X \rightarrow W$ we get a map $\theta_W :M \rightarrow W$ such that $\theta_W \circ \alpha_U =0$ and $\theta_W \circ \alpha_X = \iota_W \circ \pi_X$. By construction we have $\pi_Y \circ \theta_Y \circ \alpha_X = 0 = \pi_W \circ \iota_W \circ \pi_X = \pi_W \circ \theta_W \circ \alpha_X.$ Similarly $\pi_W \circ \theta_W \circ \alpha_U = 0 = \pi_Y \circ \iota_Y \circ \pi_Y = \pi_Y \circ \theta_Y \circ \alpha_U.$ Since $[\alpha_U \alpha_X] : U \oplus X \rightarrow M$ is surjective, the maps $\pi_W \circ \theta_W \circ [\alpha_U \alpha_X]$ and $\pi_Y \circ \theta_Y \circ [\alpha_U \alpha_X]$ being equal implies $\pi_Y \circ \theta_Y = \pi_W \circ \theta_W$. Thus, since $N$ is the pullback of $\pi_W$ and $\pi_Y$ we get a map $\theta : M\rightarrow N$ such that $\beta_W \circ \theta = \theta_W$ and  $\beta_Y \circ \theta = \theta_Y$. \\
Clearly $\theta$ satisfies the required four equations in the lemma, and since $\begin{bmatrix} \beta_W \\ \beta_Y \end{bmatrix}:N \rightarrow W\oplus Y$ is injective and $[\alpha_U \alpha_X] : U \oplus X \rightarrow M$ is surjective, $\theta$ is uniquely determined. \\
We have $\begin{bmatrix} \beta_W \\ \beta_Y \end{bmatrix} \circ \theta \circ [\alpha_U \alpha_X] = \begin{bmatrix} 0 &  \iota_W \pi_X \\ \iota_Y \pi_U & 0 \end{bmatrix} = \begin{bmatrix} 0 &  \iota_W \\ \iota_Y  & 0 \end{bmatrix}  \begin{bmatrix}\pi_U &0 \\ 0 & \pi_X \end{bmatrix} $ which implies exactness of the sequence.\\
This map has image isomorphic to $C\oplus B$ and using injectivity and surjectivity, $\theta$ has the same (up to isomorphism) image inside N. But $dim_\mathbb{F}M =dim_\mathbb{F}U +dim_\mathbb{F}X -dim_\mathbb{F}A =dim_\mathbb{F}(C\oplus B)+dim_\mathbb{F}A,$ so the sequence is exact at M, and similarly it is exact at N.
\end{proof}

\begin{definition}
Given representations $A,D\in Rep_{\mathbb{F}}Q$ write $Ext^2(D,A)$ for the set of equivalence classes of sequences $$\xi : 0 \rightarrow A \rightarrow E_1 \rightarrow E_2 \rightarrow D \rightarrow 0 \\ \xi ': 0 \rightarrow A \rightarrow E_1' \rightarrow E_2' \rightarrow D \rightarrow 0$$ where the sequences are equivalent if there exists a commutative diagram: 

$$\begin{tikzcd}
            0 \arrow{r}  & A \arrow{r} \arrow[d, equal] & E_1 \arrow{r} & E_2 \arrow{r} &D \arrow{r} & 0  \\ 
	0 \arrow{r}  & A \arrow{r} \arrow[d, equal] & E_1'' \arrow{r} \arrow{d} \arrow{u} & E_2'' \arrow{r} \arrow{d} \arrow{u} &D \arrow{r} \arrow[d, equal] \arrow[u,equal] & 0  \\ 
            0 \arrow{r}  & A \arrow{r}  & E_1' \arrow{r} & E_2' \arrow{r} &D \arrow{r} & 0   
            \end{tikzcd}$$
As with $Ext^1$ we may define an addition on $Ext^2$ as follows: given $\xi, \xi '$ consider the commutative diagram:
$$\begin{tikzcd}
            0 \arrow{r}  & A \oplus A \arrow{r} \arrow[d, equal] & E_1 \oplus E_1' \arrow{r} & E_2 \oplus E_2' \arrow{r} &D\oplus D \arrow{r} & 0  \\ 
	0 \arrow{r}  & A \oplus A \arrow{r} \arrow{d}{(1,1)}  \arrow[dr, phantom, "\ulcorner", very near start] & E_1 \oplus E_1'  \arrow{r} \arrow{d} \arrow[u, equal] & C \arrow{r} \arrow[d,equal] \arrow{u}  \arrow[ur, phantom, "\urcorner", very near end] &D \arrow{r} \arrow[d, equal] \arrow{u}{ \Delta } & 0  \\ 
            0 \arrow{r}  & A \arrow{r}  & B \arrow{r} & C \arrow{r} &D \arrow{r} & 0   
            \end{tikzcd}$$
The bottom row gives the sequence $\xi + \xi '.$
\end{definition}

\begin{exercise} 
\begin{enumerate}
\item Show that this gives a well-defined abelian group structure on $Ext^2(D,A)$ with unit $$0 \rightarrow A \xrightarrow{1} A \xrightarrow{0} D \xrightarrow{1} D \rightarrow 0$$
	Idea: Show that $Ext^2(-,-)$ is a bifunctor where addition defined above is compatible with additive structure of Hom-spaces. 
\item Show there is a morphism of abelian groups \[Ext^1(D,I) \otimes Ext^1(I,A) \rightarrow Ext^2(D,A)  \] \[  ([0 \rightarrow I  \rightarrow E_2 \rightarrow D \rightarrow 0],[0 \rightarrow A  \rightarrow E_1 \rightarrow I \rightarrow 0]) \mapsto [
            0 \rightarrow A \rightarrow E_1 \rightarrow E_2  \rightarrow D \rightarrow 0 ].\]
\item Generalize this to $Ext^n$.
\end{enumerate} 
\end{exercise} 

\begin{theorem}[Schanuel]
Let $ \xi : 0 \rightarrow A \xrightarrow{\theta_1} E_1 \xrightarrow{\theta_2} E_2 \xrightarrow{\theta_3} D \rightarrow 0 $ be a sequence in $Ext^2(D,A)$. Let $I=im(\theta_2)$. Then $\xi =0$ in $Ext^2(D,A)$ if and only if there exists a commutative diagram: 
$$\begin{tikzcd}
            &&0 \arrow{d}&0\arrow{d}& \\ 
	0 \arrow{r}  & A \arrow{r} \arrow[d,equal]  & E_1 \arrow{r} \arrow{d} \arrow[dr, phantom, "\ulcorner", very near start  ]  \arrow[dr, phantom, "\lrcorner", very near end  ]& I \arrow{r} \arrow[d]   &0  \\ 
           0 \arrow{r}  & A \arrow{r}  & V \arrow{r} \arrow{d}  & E_2 \arrow{r} \arrow[d]   &0  \\ 
	&&D \arrow[r,equal] \arrow{d}&D\arrow{d}&\\
          &&0&0&
            \end{tikzcd}$$
Which gives the sequence $$ 0 \rightarrow Hom(D,A) \rightarrow Hom(D,E_1) \rightarrow Hom(D,I) \rightarrow Ext^1(D,A) \rightarrow Ext^1(D,E_1) \rightarrow Ext^1(D,I) \rightarrow Ext^2(D,A) $$
\end{theorem}

\begin{corollary}
We get a commutative diagram:
 \[\begin{tikzcd}
             &  0\arrow{d}& 0\arrow{d} &0\arrow{d} &  \\
	0 \arrow{r} & A \arrow{r}{\iota_X} \arrow{d}{\iota_U}& X\arrow{d} \arrow{r}{\pi_X}& C\arrow{r}\arrow{d}{\iota_W}& 0 \\
	0\arrow{r}& U\arrow{d}{\pi_U} \arrow{r} & V \arrow{r} \arrow{d} & W\arrow{r} \arrow{d}{\pi_W}& 0 \\
           0 \arrow{r} & B \arrow{r}{\iota_Y} \arrow{d}& Y \arrow{d} \arrow{r}{\pi_Y}& D\arrow{r}\arrow{d}& 0 \\
	& 0 & 0& 0&  
            \end{tikzcd}\]
\end{corollary}

\begin{proof}
Since $Ext^2(D,A)=0$ write $I=im(\theta)$, there exists a commutative diagram:

$$\begin{tikzcd}
            &&0 \arrow{d}&0\arrow{d}& \\ 
	0 \arrow{r}  & A \arrow{r} \arrow[d,equal]  & M \arrow{r} \arrow{d}{\omega_M} \arrow{dr}{\theta} & I \arrow{r} \arrow[d]   &0  \\ 
           0 \arrow{r}  & A \arrow{r}  & V \arrow{r}{\omega_N} \arrow{d}  & N \arrow{r} \arrow[d]   &0  \\ 
	&&D \arrow[r,equal] \arrow{d}&D\arrow{d}&\\
          &&0&0&
            \end{tikzcd}$$
Define maps:
\begin{align*} 
\omega_M \circ \alpha_U &: U \rightarrow V  &  \beta_W \circ \omega_N &: N \rightarrow W    \\
 \omega_M \circ \alpha_X &: X \rightarrow V   &   \beta_Y \circ \omega_N &: V \rightarrow Y
\end{align*}
 By definition we have: 
\begin{align*} 
(\omega_M \circ \alpha_U) \circ \iota_U &= (\omega_M \circ \alpha_X) \circ \iota X  \\
(\beta_Y \circ \omega_N) \circ (\omega_N \circ \alpha_U) &= \beta_Y \circ \theta \circ \alpha_U = \iota_Y \circ \pi_U \\
(\beta_W \circ \omega_N) \circ (\omega_M \circ \alpha_X) &= \beta_W \circ \theta \circ \alpha_X = \iota_W \circ \pi_X \\
\pi_W \circ (\beta_W \circ \omega_N)  &=  \pi_Y \circ  (\beta_Y \circ \omega_N)
\end{align*} 
Thus the diagram commutes. It remains to show the middle row and column are exact. \\
By construction $\alpha_U, \alpha_X,$ and $\omega_M$ are injective and $\beta_W, \beta_Y,$ and $\omega_N$ are surjective. \\
We also have:
\begin{align*} 
(\beta_W \circ \omega_N) \circ (\omega_M \circ \alpha_U) &= \beta_W \circ \theta \circ \alpha_U &= 0 \\
(\beta_Y \circ \omega_N) \circ (\omega_M \circ \alpha_X) &= \beta_Y \circ \theta \circ \alpha_X &= 0
\end{align*}
So $im(\omega_M \circ \alpha_{\bullet}) \subseteq ker(\beta_{\bullet} \circ \omega_N)$ with $\bullet = X$ and $Y$ or $\bullet = U$ and $W$. 
But 
\begin{align*} 
dim_{\mathbb{F}}V &= dim_{\mathbb{F}}M + dim_{\mathbb{F}}D \\
&=dim_{\mathbb{F}}U +dim_{\mathbb{F}}X - dim_{\mathbb{F}}A +dim_{\mathbb{F}}D \\
&= dim_{\mathbb{F}}B +dim_{\mathbb{F}}X + dim_{\mathbb{F}}D \\
&= dim_{\mathbb{F}}X + dim_{\mathbb{F}}Y
\end{align*}
So the middle column is exact. A similar calculation gives that the middle row is exact
\end{proof}


%4/9/18 notes 
\begin{definition}
        Let $Q$ be an acyclic quiver. The \textbf{Auslander-Reiten quiver} $\Gamma$ of $Q$ has as vertices the isomorphism classes of indecomposable representations and has an arrow $V\to W$ if there exists an irreducible morphism $\theta:V\to W.$
    \end{definition}
    
    \begin{definition}
        Given a sink-adapted sequence $(i_1,\ldots,i_n)$ of $Q$, there is a Coxeter functor $C_+=\Sigma_{i_n}^+\cdots\Sigma_{i_1}^+$ acting on $\rep_\mathbb{F}Q.$
    \end{definition}
    
    \begin{theorem}[Brenner-Butler]
        The Coxeter functor $C_+$ identifies with the Auslander-Reiten translation $\tau$ acting on $\rep_\mathbb{F}Q.$
    \end{theorem}
    
    We've seen the following properties of $\tau$:
    \begin{enumerate}
        \item If $V$ is indecomposable and not projective, then $\tau(V)$ is indecomposable. This implies that $\tau$ acts on vertices of the Auslander-Reiten quiver $\Gamma$ of $Q$.
        \item $V$ is projective (resp. injective) if and only if $\tau(V)=0$ (resp. $\tau^{-1}(V)=0$). If $V$ is not projective, there exists a nonsplit exact sequence 
        \begin{tikzcd}
        0\arrow{r}&\tau(V) \arrow{r} &X\arrow{r}&V\arrow{r}&0
        \end{tikzcd}.
        \item If $V,W\in \rep_\mathbb{F}Q$ have no projective summands, then $\Hom_Q(V,W)\cong \Hom_Q(\tau V,\tau W)$.
        \item A representation $V$ is preprojective (resp. preinjective) if $\tau^k(V)=0$ for some $k>0$ (resp. $k<0$). For a Dynkin quiver, every (indecomposable) representation is preprojective. 
    \end{enumerate}
    
    Last time we showed that $\theta:V\to W$ is irreducible if and only if $\theta\in \text{Rad}_Q(V,W)\setminus \text{Rad}^2_Q(V,W)$.
    
    \begin{corollary}
        The Auslander-Reiten quiver $\Gamma$ has no loops, i.e. there is no indecomposable representation $V$ with an irreducible endomorphism.
    \end{corollary}
    
    \begin{proof}
        Let $\theta:V\to W$ be a non-isomorphism. Then it factors as $V\twoheadrightarrow \im\theta \hookrightarrow V$, an element of $\text{Rad}^2_Q(V,V)$. So $\theta$ is not ireeducible.
    \end{proof}
    
    \begin{lemma}
        Let $Q$ be a Dynkin quiver. For indecomposable representations $V$ and $W$, the following hold:
        \begin{enumerate}
            \item $\Ext_Q(U,W)$ and $\Ext_Q(W,U)$ cannot both be nonzero.
            \item $\Hom_Q(W,U)$ and $\Ext_Q(W,U)$ cannot both be nonzero.
            \item $\Hom_Q(U,W)$ and $\Hom_Q(W,U)$ cannot both be nonzero.
        \end{enumerate}
    \end{lemma}
    
    \begin{proof}
        Let $(i_1,\ldots,i_n)$ be a sink-adapted sequence for $Q$ and $\tau=\Sigma_{i_n}^+\cdots \Sigma_{i_1}^+$. Then we can write 
        \begin{align*}
            U&\cong \tau^{-k}\Sigma_{i_1}^-\cdots\Sigma_{i_{r-1}}^-S_{i_r}\\
            W&\cong\tau^{-l}\Sigma_{i_1}^-\cdots\Sigma_{i_{t-1}}^-S_{i_t}
        \end{align*}
        for $k,l\geq 0$ and $1\leq r,t\leq n$. Without loss of generality, assume $k\leq l$. Then \[\Ext_Q(U,W)\cong\Ext_Q(\Sigma_{i_1}^-\cdots\Sigma_{i_{r-1}}^-S_{i_r},\tau^k W)=0.\] In the same way, we see that $\Ext_Q(W,U)=0$ when $k=l$. So assume $k<l$. Then \[\Hom_Q(W,U)\cong\Hom_Q(\tau^k,\Sigma_{i_1}^-\cdots\Sigma_{i_{r-1}}^-S_{i_r}).\] Consider $\theta:\tau^kW\to\tau^kU$. Since $\tau^kU$ is projective, $\im\theta\subseteq\tau^kU$ is also projective. But then the map $\tau^kW\twoheadrightarrow\im\theta$ must split, which implies that $\tau^k\cong\im\theta$ by indecomposability. But $\tau^kW$ is not projective (since $k<l$), a contradiction. So we must have $\theta=0$.
    \end{proof}
    
    \underline{Idea}: Draw the Auslander-Reiten quiver so that $W$ lies to the right of $U$ if there exists a nonzero morphism $\theta:U\to W$. Then ``Homs go to the right and Exts go to the left."
    
    \begin{example}
        $Q=1\leftarrow 2\leftarrow\cdots\leftarrow n$. Write $V_{i,j}$, $i\leq j$, for the representation with one-dimensional spaces at vertices $i\leq k\leq j$ and identity maps when this makes sense. These are all the indecomposables. The A-R quiver of $Q$ has the general pattern:
        \[\begin{tikzcd}
            &V_{i-1,j+1}\arrow{dr}&\\
            V_{i-1,j}\arrow{ur}\arrow{dr}&&V_{i,j+1}\arrow{ll}{\tau}\\
            &V_{i,j}\arrow{ur}&
        \end{tikzcd}\]
    \end{example}
    
    \begin{lemma}
        Let $Q$ be an acyclic quiver and $V,W\in\rep_\mathbb{F}Q$ indecomposable representations with $\tau V, \tau W\neq 0$. There exists an irreducible morphism $\theta:V\to W$ if and only if there exists an irreducible morphism $\phi:\tau V\to\tau W.$
    \end{lemma}
    
    \underline{Idea of proof}: If $\tau\theta$ is not irreducible, then it factors as $\phi_2\circ\phi_1:\tau V\to Z\to\tau W$. Since $Z$ has no injective summands, $\theta$ factors as $\tau^{-1}\phi_2\circ \tau^{-1}\phi_1.$
    
    % 4/11/18
    
    \begin{prop}
        The Auslander-Reiten quiver $\Gamma$ of a Dynkin quiver $Q$ has no oriented cycles. 
    \end{prop}
    
    \begin{proof}
        Suppose there exists an oriented cycle $V_1\to V_2\to \cdots\to V_k \to V_1$ in $\Gamma$ where each $V_i$ is indecomposable and each map is irreducible. Apply $\tau$ the minimal number of times so that some $V_i$ becomes projective, say $V_1$. Then the map $V_k\to V_1$ must be a monomorphism with $V_k$ projective. By induction, each $V_i$ is projective and each map must be a monomorphism. Then composing we get an injective map $V_1\to V_1$ which must be an isomorphism. It follows that all maps $V_i\to V_{i+1}$ are isomorphisms, contradicting irreducibility. 
    \end{proof}
    
    \begin{lemma}
        Let $Q$ be a Dynkin quiver. For any positive roots $\alpha,\alpha'\in\Phi^+$, there exists $h_{\alpha,\alpha'}\in\mathbb{N}$ so that $\dim_\mathbb{F}\Hom_Q(V(\alpha,\mathbb{F}),V(\alpha',\mathbb{F}))=h_{\alpha,\alpha'}$ is independent of the field $\mathbb{F}$.
    \end{lemma}
    
    \begin{proof}
        If $\alpha=\alpha'$, we know that $V(\alpha,\mathbb{F})$ is rigid and thus $1=q(\alpha)=\left<\alpha,\alpha\right>=\dim_\mathbb{F}\Hom_Q(V(\alpha,\mathbb{F}),V(\alpha,\mathbb{F}))-\dim_\mathbb{F}\Ext_Q(V(\alpha,\mathbb{F}),V(\alpha,\mathbb{F}))=\dim_\mathbb{F}\Hom_Q(V(\alpha,\mathbb{F}),V(\alpha,\mathbb{F}))-0$. So $h_{\alpha,\alpha}=1$. If $V(\alpha,\mathbb{F})$ is projective, say it is the projective cover of the simple $S_i$. Then $\dim_\mathbb{F}\Hom_Q(V(\alpha,\mathbb{F}),V(\alpha',\mathbb{F}))=a_i'$ is independent of $\mathbb{F}.$ If $V(\alpha,\mathbb{F})$ is not projective but $V(\alpha',\mathbb{F})$ is projective, then $\dim_\mathbb{F}\Hom_Q(V(\alpha,\mathbb{F}),V(\alpha',\mathbb{F}))=0.$ Otherwise, we can apply $\tau$ to reduce to one of these cases.
    \end{proof}
    
    \begin{corollary}
        Let $\beta,\beta':\Phi^+\to\mathbb{N}$ be multisets of positive roots. There exists $h_{\beta,\beta'}\in\mathbb{N}$ so that $\dim_\mathbb{F}\Hom_Q(V(\beta,\mathbb{F}),V(\beta',\mathbb{F}))=h_{\beta,\beta'}$ is independent of $\mathbb{F}$.
    \end{corollary}
    
    \begin{proof}
        $\Hom_Q(V(\beta,\mathbb{F}),V(\beta',\mathbb{F}))\cong\bigoplus_{\alpha,\alpha'}\Hom_Q(V(\alpha,\mathbb{F}),V(\alpha',\mathbb{F}))^{\oplus\beta(\alpha)\beta'(\alpha')}$. So $h_{\beta,\beta'}=\sum_{\alpha,\alpha'\in\Phi^+}\beta(\alpha)\beta'(\alpha')h_{\alpha,\alpha'}$. 
    \end{proof}
    
    \begin{corollary}
        For positive roots $\alpha,\alpha'\in\Phi^+$, there exists $e_{\alpha,\alpha'}\in\mathbb{N}$ so that  $\dim_\mathbb{F}\Ext_Q(V(\alpha,\mathbb{F}),V(\alpha',\mathbb{F}))=e_{\alpha,\alpha'}$ is independent of $\mathbb{F}$. Similarly for $\beta,\beta':\Phi^+\to\mathbb{N}$.
    \end{corollary}
    
    \begin{proof}
        \begin{align*}
        \dim_\mathbb{F}\Ext_Q(V(\alpha,\mathbb{F}),V(\alpha',\mathbb{F}))&=\dim_\mathbb{F}\Hom_Q(V(\alpha,\mathbb{F}),V(\alpha',\mathbb{F}))-\left<\alpha,\alpha'\right>\\
        &=h_{\alpha,\alpha'}-\left<\alpha,\alpha'\right> \\&=: e_{\alpha,\alpha'}.
        \end{align*}
    \end{proof}
    
    \begin{definition}
        Let $Q$ be a Dynkin quiver and $\Phi^+$ the associated set of positive roots. A total ordering $\prec$ on $\Phi^+$ is \textbf{$Q$-admissible} if $h_{\alpha,\alpha'}\neq 0$ implies $\alpha\preceq\alpha'.$
    \end{definition}
    
    \begin{lemma}
        A total ordering $\prec$ on $\Phi^+$ is $Q$-admissible if and only if $\left<\alpha,\alpha'\right>>0$ implies $\alpha\preceq\alpha'.$ Such an ordering also satisfies: $\left<\alpha,\alpha'\right><0$ implies $\alpha'\prec\alpha.$
    \end{lemma}
    
    \begin{proof}
        Recall that $\Hom_Q(V(\alpha,\mathbb{F}),V(\alpha',\mathbb{F}))$ and $\Ext_Q(V(\alpha,\mathbb{F}),V(\alpha',\mathbb{F}))$ cannot both be nonzero. So $\left<\alpha,\alpha'\right>>0$ if and only if $h_{\alpha,\alpha'}\neq 0$ and $\left<\alpha,\alpha'\right><0$ implies $e_{\alpha,\alpha'}>0$ and $h_{\alpha,\alpha'}=0.$
    \end{proof}
    
    \begin{definition}
        Let $Q$ be Dynkin and $(i_1,\ldots,i_n)$ a sink adapted sequence. Define a total ordering $\prec_Q$ on $\Phi^+$ as follows: For $\alpha,\alpha'\in\Phi^+$, write $V(\alpha,\mathbb{F})\cong\tau^{-k}\Sigma_{i_1}^-\cdots\Sigma_{i_{r-1}}^-S_{i_r}$ and $V(\alpha',\mathbb{F})\cong\tau^{-l}\Sigma_{i_1}^-\cdots\Sigma_{i_{t-1}}^-S_{i_t}$ for $k.l\geq 0$ and $1\leq r,t\leq n$. Then write $\alpha\prec_Q\alpha'$ if $k<l$ or $k=l$ and $r<t$.
    \end{definition}
    
    \begin{lemma}
        The total ordering $\prec_Q$ on $\Phi^+$ is $Q$-admissible.
    \end{lemma}
    
    \begin{proof}
        Recall for indecomposable representations $U,W\in\rep_\mathbb{F}Q$ with $\tau U,\tau W\neq 0$ that $\Hom_Q(\tau U,\tau W)\cong\Hom_Q(U,W).$ Consider $\alpha,\alpha'\in\Phi^+$ with $\alpha'\prec_Q\alpha$. We need to show $h_{\alpha,\alpha'}=0.$ Indeed, 
        \begin{align*}
            h_{\alpha,\alpha'}&=\dim_\mathbb{F}\Hom_Q(V(\alpha,\mathbb{F}),V(\alpha',\mathbb{F}))\\
            &=\dim_\mathbb{F}\Hom_Q(\tau^lV(\alpha,\mathbb{F}),\Sigma_{i_1}^-\cdots\Sigma_{i_{t-1}}^-S_{i_t})
        \end{align*}
        If $l<k$, then $\tau^l V(\alpha,\mathbb{F})$ is not projective and so $h_{\alpha,\alpha'}=0.$ So assume $l=k$ so that \[h_{\alpha,\alpha'}=\dim_\mathbb{F}\Hom_Q(\Sigma_{i_1}^-\cdots\Sigma_{i_{r-1}}^-S_{i_r},\Sigma_{i_1}^-\cdots\Sigma_{i_{t-1}}^-S_{i_t})\] with $t<r$. But then \[h_{\alpha,\alpha'}=\dim_\mathbb{F}\Hom_Q(\Sigma_{i_t}^-\cdots\Sigma_{i_{r-1}}^-S_{i_r},S_{i_t})\] where $S_{i_t}$ is a simple projective for $Q'$ and so $h_{\alpha,\alpha'}=0.$
     \end{proof}
     
     
     %4/13/18
    
    \vspace{1cm} \underline{Last Time}: 
    \begin{enumerate}
        \item For a Dynkin quiver $Q$ and multisets $\beta,\beta':\Phi^+\to\mathbb{N}$, there exists a polynomial $\eta_{\beta,\beta'}(T)=T^{h_{\beta,\beta'}}\in\mathbb{Z}[T]$ so that $|\Hom_Q(V(\beta,\mathbb{F}),V(\beta',\mathbb{F}))|=\eta_{\beta,\beta'}(|\mathbb{F}|)$ for any finite field $\mathbb{F}$. 
        \item There exists a total ordering $\prec_Q$ on $\Phi^+$ so that $h_{\alpha,\alpha'}\neq 0$ implies $\alpha\prec_Q\alpha'$.
    \end{enumerate}
    
    For $\alpha\in\Phi^+$ and $n\in\mathbb{N}$, define $a_{n\alpha}\in\mathbb{Z}[T]$ by \[a_{n\alpha}(T):= \prod_{i=1}^n (\eta_{\alpha,\alpha}^n-\eta_{\alpha,\alpha}^{i-1}).\]
    For a multiset  $\beta:\Phi^+\to\mathbb{N}$, define $a_\beta\in\mathbb{Z}[T]$ by \[a_{\beta}(T):= \prod_{\alpha\in\Phi^+}a_{\beta(\alpha)\alpha}\cdot\prod_{\substack{\alpha,\alpha'\in\Phi^+ \\ \alpha\neq\alpha'}}\eta_{\alpha,\alpha'}^{\beta(\alpha)\beta(\alpha')}\]
    Note: Each $a_\beta$ is a monic polynomial. 
    
    \begin{lemma}
        $|\Aut_Q(V(\beta,\mathbb{F}))|=a_\beta(|\mathbb{F}|)$ for any finite field $\mathbb{F}$.
    \end{lemma}
    
    \begin{proof}
        Since $V(\beta,\mathbb{F})=\bigoplus_{\alpha\in\Phi^+}V(\alpha,\mathbb{F})^{\oplus\beta(\alpha)}$, we may use the total ordering $\prec_Q$ to view an automorphism of $V(\beta,\mathbb{F})$ as an upper triangular matrix with entries indexed by $\Phi^+\times\Phi^+$ whose entry in position $(\alpha,\alpha')$ is zero if $\alpha'\prec_Q\alpha$, is an arbitrary element of $\Hom_Q(V(\alpha,\mathbb{F})^{\oplus\beta(\alpha)},V(\alpha',\mathbb{F})^{\oplus\beta(\alpha')})$ for $\alpha\prec_Q\alpha'$, and is an automorphism of $V(\alpha,\mathbb{F})^{\oplus\beta(\alpha)}$ if $\alpha=\alpha'$. From above, we have $|\Hom_Q(V(\alpha,\mathbb{F})^{\oplus\beta(\alpha)},V(\alpha',\mathbb{F})^{\oplus\beta(\alpha')})|=\eta_{\alpha,\alpha'}^{\beta(\alpha)\beta(\alpha')}(|\mathbb{F}|).$ On the other hand, $\End_Q(V(\alpha,\mathbb{F}))\cong\mathbb{F}$ and so $\Aut_Q(V(\alpha,\mathbb{F})^{\oplus\beta(\alpha)})\cong\text{GL}_{\beta(\alpha)}(\mathbb{F})$ which has cardinality $a_\beta(|\mathbb{F}|)$.
    \end{proof}


  \section{Relation to Symmetric Functions}

  \section{Ringel's Theorem}


%%%%%%%%%%%%%%%%%%%%%%%%%%%%%%%%%%%%%%%%%%%%%%%%%%%%%%%%%%%
\chapter{Canonical Bases for Quantized Enveloping Algebras}

  \section{Lusztig Quiver Varieties}

  \section{IC Sheaves}

  \section{Crystal Bases}


%%%%%%%%%%%%%%%%%%%%%%%%%%%%%%%%%%%%%%%%%%%%%%%%%%%%%%%%%%%%%%%%
\chapter{Representation Theory of Quantized Enveloping Algebras}


%%%%%%%%%%%%%%%%%%%%%%
\chapter{Applications}

  \section{Integrable Lattice Models}

  \section{Knot Theory}



\end{document}
